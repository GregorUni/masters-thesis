\chapter{Background}
\label{sec:state}

% Hier werden zwei wesentliche Aufgaben erledigt:

% 1. Der Leser muß alles beigebracht bekommen, was er zum Verständnis
% der späteren Kapitel braucht. Insbesondere sind in unserem Fach die
% Systemvoraussetzungen zu klären, die man später benutzt. Zulässig ist
% auch, daß man hier auf Tutorials oder Ähnliches verweist, die hier auf
% dem Netz zugänglich sind.

% 2. Es muß klar werden, was anderswo zu diesem Problem gearbeitet
% wird. Insbesondere sollen natürlich die Lücken der anderen klar
% werden. Warum ist die eigene Arbeit, der eigene Ansatz wichtig, um
% hier den Stand der Technik weiterzubringen? Dieses Kapitel wird von
% vielen Lesern übergangen (nicht aber vom Gutachter ;-), auch später
% bei Veröffentlichungen ist "Related Work" eine wichtige Sache.

% Viele Leser stellen dann später fest, daß sie einige der Grundlagen
% doch brauchen und blättern zurück. Deshalb ist es gut,
% Rückwärtsverweise in späteren Kapiteln zu haben, und zwar so, daß man
% die Abschnitte, auf die verwiesen wird, auch für sich lesen
% kann. Diese Kapitel kann relativ lang werden, je größer der Kontext
% der Arbeit, desto länger. Es lohnt sich auch! Den Text kann man unter
% Umständen wiederverwenden, indem man ihn als "Tutorial" zu einem
% Gebiet auch dem Netz zugänglich macht.

% Dadurch gewinnt man manchmal wertvolle Hinweise von Kollegen. Dieses
% Kapitel wird in der Regel zuerst geschrieben und ist das Einfachste
% (oder das Schwerste weil erste).

% Background Warum ist das Thema wichtig?
% Was ist das Smart Grid -> mit Background verknüpfen
% Vor, Nachteile vom Smart Grid?
% Was sind Smart Meter
% Warum sind Smart Meter Privacy relevant?
% DC-Netze einführen
% Related Work: Welche andere Verfahren gibt es?
% Vllt erklärung regeln vom BSI

This section introduces an overview of the basic concepts for this work. Therefore, the key components of the smart grid are explained, what structural changes and what challenges the smart grid will bring.
In addition, this chapter discusses the current state of research.\\

\section{Smart Grid}
The original energy network was mainly considered as a transmission system to send electricity from the generators via a elongated network of cables and transformers to the consumers.% vllt hier noch schreiben, dass es erneuerbare Energien wegen der Klimakriese gibt
Instead of a few electricity producers (e.g. nuclear power plants, coal-fired power plants), which were responsible for a large part of the electricity generation, there are now many smaller producers (e.g. wind turbines). %However, renewable power generation is often dependent on external environmental influences. Therefore, smart meters have been widely introduced in households to ensure that the smart grid is stable despite fluctuations in power generation.
But, renewable power generation is often dependent on external environmental factors. In order for the electricity grid to be stable despite fluctuations in power generation, smart meters have been introduced.
Smart Meters enable the electricity provider to receive the electricity consumption of a household every 15 minutes. It offers the possibility to get more easily the current electricity demand from the consumers. Previously, the current electricity demand was simulated from load forecasting models. If the demand should increase spontaneously, peaker plants, mainly consisting of coal-fired power plants, would be turned on to quickly meet this demand. This is costly and environmentally unfriendly. 
Since then, structural changes have been made to optimize the energy grid and make it more intelligent by exchanging information in near-real-time. This allows the demand to be matched to the available supply. The fundamental component of the smart grid are the smart meters which will be discussed in more detail in the next section.(Quelle:Smart Grid Communications)(Privacy Survey2013)\begin{figure}[tbp]
  \centering
  \includegraphics[width=0.8\textwidth]{images/Smart_Grid.png}
  \caption[Short description]{An example of a NILM analysis.}
  \label{fig:Appliance_Model}
\end{figure}
\\
\\
\textbf{Smart Meter}
\\
\\
Smart meters are the key component in a smart grid. A smart meter is an electricity meter which has an interface to the Internet. The additional functions that a smart meter brings to a regular electricity meter allow a two-way communication between the control center and the smart meter. The interconnection of electricity meters and control center via the Internet is also called Advanced Metering Infrastructure (AMI). The resulting communication between both components improves the quality of the power grid and makes it possible to offer services that would not be feasible without a smart meter. For example it's now practicable to detect power outages for a grid operator on its own.%hier nochmal was ändern.
Previously, the operator was dependent on customer calls to detect power outages. Another new feature is detailed monitoring of power flows at the smart meter. Beforehand, power flows could only be measured up to substations. Moreover, the advanced functions enables electricity network operators to quickly detect changes in consumption behavior and react to them without having to use peaker plants, which are costly and environmentally unfriendly. Depending on the setting, smart meters can send electricity consumption to the electricity provider at least every 15 minutes. Additionally, in combination with the consumption of all users and the current electricity supply, real-time pricing becomes possible. Not only the customer can be offered a better electricity contract, the smart meters no longer have to be read at home by a technician from the electricity provider. As a result billing becomes easier for customers and electricity providers. Furthermore, customers can also check their current electricity consumption via the interfaces provided by the smart meter in order to analyze their own behavior and to reduce their consumption. (Privacy-Aware Smart Metering)
%Thus, the end user can analyze his own consumption behavior in order to reduce his own electricity consumption.
\section{Smart Meter Privacy}
The main advantage of the smart grid is the advanced communication between the consumers smart meter and the energy suppliers. The 15-minute messages from the electricity meter provide the electricity supplier with a regular update on the status of the electricity grid and there is no longer any need to rely on forecasting models. However, sending user information in such a short period of time allows for new methods that can be used to create accurate behavioral analyses in one's own home. Sending private electricity consumption data is therefore very sensitive information and has to be protected. This is not an easy task, because on the one hand the electricity consumption must be protected and anonymized, and on the other hand the billing and costs must be clearly assignable to a person. The two problems are referred to as metering for billing and metering for operations. The following paragraph describes how simple behavioral analyses are generated by electricity consumption. Subsequently, solutions to Metering for Billing and Metering for Operations will be presented, which have been discussed in the scientific community so far.(Privacy-Aware Smart Metering) \begin{figure}[htbp!]
  \centering
  \includegraphics[width=0.9\textwidth]{images/Appliance_Model.png}
  \caption[Short description]{An example of a NILM analysis.}
  \label{fig:Appliance_Model}
\end{figure}
\subsection{Non-intrusive load monitoring} 

Interpreting power consumption with the intent of identifying devices in the home is called non-intrusive load monitoring (NILM). George Hart and Fred Schweppe were the first to develop non-intrusive load monitors in 1985 and connected them to electricity meters. They were able to record the current power consumption up to every 5 seconds. Then they did the following steps to identify appliances in a household:
\begin{enumerate}
\item Edge Detection:\\
First, the intercepted electricity consumption was stored. Afterwards, a search for strongly rising or strongly falling edges was performed. These edges indicate that a device may have been switched on or off at that moment.
\item Cluster Analysis:\\ 
The stored events of steeply rising or steeply falling edges are visualized in a graph with the following characteristics. Each event is ordered according to how much power was consumed or how much power was 	``released'' from the device (e.g. when it was switched off). 
Essentially, a cluster analysis is then applied to the diagram and each  cluster represents a household appliance.
\item Appliance Model Construction\\
Since different household appliances have been determined by the clusters, appliance models can now be constructed. In this step, different states in which an appliance can be in, are found based on the different power consumption. An example of how the result of a appliance model looks like can be seen in Figure 2.1.
\item Behavior Analysis:\\
Once the majority of the household appliances have been identified, the behaviors of people in the household can be analyzed. In real time, it is possible to track the use of devices, since individual signals can be identified as they occur and do not need to be reconstructed anymore.
At this point, several approaches can be taken to provide behavioral analysis. A common approach is to track how long a device has been in use and create statistics on how each device has been used. A daily analysis can be viewed in Figure 2.2.
\item Appliance Saving:\\
The last approach is to name the household appliances found(washing machine, etc.) and store them in a database. So that in the case of a further household analysis, it is possible to fall back on appliances that have already been found.
\end{enumerate}
The founder of NILM G. W. Hart himself said in 1989: ``Specifically, I recommend that legal restrictions be enacted or clarified so that electric power usage is considered as private as any phone conversation.''(Residential Energy Monitoring)
Through Nilm, simple observations can be made without analyzing the household behavior for a longer time. For example, it can be noticed when no one is at home because no lamps are on. It can also be quickly assumed that the house inhabitants are on vacation, if the power consumption is lower than usual over days. For burglars, this information would be particularly useful, as they would have no problem knowing when is a suitable time to break in.
\begin{figure}[tbp]
  \centering
  \includegraphics[width=1\textwidth]{images/nilm.png}
  \caption[Short description]{An example day of a NILM analysis.}
  \label{fig:Nilm}
\end{figure}
\\
\\
\textbf{High Resolution Analysis}
\\
\\
Since then, research in the field of intrusive monitoring has continued. It was investigated both how much information can be extracted from the household through electricity consumption when electricity consumption was measured particularly frequently. Furthermore, it was investigated whether assumptions can be made about the behavior in the household even at low resolution.
For example, in the paper(Multimedia Content), the movie being watched could be determined by the power consumption of an LCD television. TV power consumption is strongly influenced by blacklighting activities and each movie has a unique brightness signature. This unique signature was exploited to make a statement about the film being viewed.
In the paper, the power consumption of the TV was measured every 2 seconds and after 5 minutes analysis of the consumption, the content of the program could be determined with high probability. For this purpose, a Power Consumption Prediction Function was trained to estimate the power consumption of the TV based on the brightness of movie sequences. The input of the prediction function is 5 minute sequences of a movie and the output is the possible power consumption of the TV. 
In order to recognize a movie, the result of the prediction function is stored over the 5 minute movie sequences. If the same 5 minute sequences are running on a TV, the power consumption can be compared with the prediction by a correlation coefficient. For multiple matches with a correlation coefficient higher than 0.85, an additional optimization algorithm is applied to estimate the movie. In the figure the result of the estimation of the prediction function is shown in green and in red you can see the actual power consumption of a LCD TV for the same movie sequence. 
\begin{figure}[tbp]
  \centering
  \includegraphics[width=1\textwidth]{images/Fernseher.png}
  \caption[Short description]{An example day of a NILM analysis.}
  \label{fig:Nilm}
\end{figure}
\\
\\
\textbf{Low Resolution Analysis}
\\
\\
Another approach is to work backwards from large data sets to obtain detailed information from aggregated electricity consumption. The Nilm method attempted to have household appliances detected from individual households and then behavioral analyses could be generated from the detected household appliances. A common approach with low-resolution electricity data is to identify the numerous factors that influence total electricity consumption and to filter them out. 
In (A Neuron Nets Based) an artificial neural network was trained to identify household appliances from electricity consumption. The power consumption was measured only every 15 minutes.\\%nochmal machen
In another work(Energy Disaggregation), the power consumption of appliances was measured only every hour. The experiment tried to disaggregate the aggregated power consumption with Discriminative Sparse Coding. The experiment was conducted as follows. First, the disaggregation algorithm was trained with electricity consumption of houses from a data set. Then, the trained algorithm was applied to the aggregated electricity consumption of 2 houses. The algorithm was expected to determine up to 52 individual household appliances from the electricity consumption. The results in the paper showed that the disaggregation algorithm with Discriminative Sparse Coding was up to 55.05 percent correct in its decisions.

\section{Related Work}
\subsection{Metering for Operations}
\\
The paragraph deals with solutions for Metering for Operations, which has been previously discussed in other scientific works. At Metering for Operations, there is currently no established consensus on a solution. Various technical proposals have already been presented in scientific papers, but there is a lack of uniform criteria and often different conditions are set for the power grid. One reason for this could be that smart grids are realized differently in other countries. In the following, the different approaches are divided into categories and presented conceptually.\\
\\
\textbf{Anonymization or Pseudonymization
Without Aggregation}
\\
\\
This approach describes the removal of smart information that allows identification. Identifiable information can also be replaced by pseudonyms. Solutions with trusted third parties are often used in this case. A trusted third party usually acts as an intermediary between the customer and the power grid provider. The trusted third party must be acknowledged by all participants and take a neutral position. In practice, however, this is difficult to achieve because the trusted third party is often hired as a service provider by the electricity supplier and is therefore also paid by the supplier.\\
\\
In the paper "A privacy-preserving Concept for Smart Grids" by Petrlic[quelle], a trusted third party is used as an intermediary. In the proposal, a smart meter communicates with a trusted third party. Certificates formed with a public key infrastructure can be used to verify and validate information flows from smart meters at the TTP. As soon as the Trusted Third Party has checked the correctness of the smart meter information, it can pseudonomize/anonymize all the necessary information. Only then is the further processed anonymized information forwarded to the electricity provider by the TTP in encrypted form. This means that the electricity provider cannot assign individual electricity consumption to its customers. With this procedure, smart meters can be anonymized. 
However, if it is possible for an attacker to record the data traffic between the smart meter and the TTP, then the attacker could forward the time stamps and smart meter identification to the electricity provider. Using these two pieces of information, the electricity provider could at least gain some insight, since it would be possible to match when information is sent to the TTP and when it is received by electricity provider.(Privacy-Aware Smart Metering) \\
\\
\textbf{Aggregation with Trusted Third Parties}
\\
\\
In the attack just described, the electricity provider tries to link two events. One is the arrival of the message at the TTP and the other is the arrival of the message at the provider itself. One way to prevent this attack is aggregation. In this case, the smart meter sends its electricity consumption to the TTP. Certificates are also sent from the smart meter so that the TTP can check the information for correctness and authenticity. Instead of forwarding the information to the electricity provider, the TTP waits until all smart meters have sent their data for which the TTP is responsible. This data is all added up and a message is sent from the TTP to the electricity provider with the total electricity consumption of all smart meters. From the aggregated value, it is not possible to extract an individual smart meter's electricity consumption, which is why the electricity provider cannot filter out information about individual customers.(A Privacy Model for Smart Metering)\\
Homomorphic encryption approaches also fall into this category. Homomorphic encryption algorithms allow simple operations such as addition and multiplication to be performed on the encrypted messages.  In some homomorphic encryption schemes, only addition OR multiplication are supported. These are then called partial homomorphic encryption.
There are also bihomorphic encryption approaches. Here not only the operations on the ciphertexts are homomorphic, but also the operations on the keys. This means that if a plaintext a is encrypted with the key x and a plaintext b is encrypted with the key z, that one can decrypt the ciphertexts enc(a+b) with the keys x+z. A bihomomorphic encryption approach with TTP has been proposed by Vetter et al.[quelle] In this case, the TTP acts as the key authority. This means that it creates all cryptographic keys and forwards them to the smart meter, which is then used for further communication with a central store. The smart meter encrypts its data and sends it to the central storage. The central storage also stores the incoming data in encrypted form, so that no unencrypted data can be found on the storage. In addition, the central storage has no access to the keys and thus has no way to decrypt the information or access meaningful data.
Therefore, the central repository has to be trusted only in terms of functionality. If an electricity provider wants to know the electricity consumption of its customers, it makes a request to the central repository, which sends the aggregated encrypted data to the electricity provider. In order for the electricity provider to decrypt the data, the key authority has to release the correct keys. It is impossible for the electricity provider to query the value of just one smart meter. This is because the key authority can only issue keys that can decrypt aggregated totals. It is guaranteed by the homomorphic encryption method which is used. The advantage of using this approach is that the different functionalities, namely storage of data and key acquisition for confidentiality and authenticity is realised from different participants.(Privacy-Aware Smart Metering)
\\
\\
\textbf{Aggregation Without a Trusted Third Party}
\\
\\
The solution proposed in this thesis is also one of the methods that aggregate without a trusted third party. The advantage of this approach is that no one has to trust a trusted third party. In general, one has to ask the questions who aggregates the data and who generates/uses the keys. In addition, a common problem to consider is how the procedure deals with a few participants/customers. \\
In the solution of Mármol et al. again a bihomomorphic encryption method is proposed. The approach of Mármol has already been discussed and implemented in a master thesis at this chair.[quelle biselli] As a reminder, bihomorphic encryption algorithms can perform simple operations such as addition on both the ciphertext and the keys. This property is exploited in the presented method of Mármol. Since it aggregates the keys and not the power consumptions as before. Furthermore, it does not matter which bihomomorphic encryption method is used, as long as all smart meters agree on one method. A key is generated from every smart meter in the power grid. Afterwards, the key is used to encrypt the power consumption. The key is then used to encrypt the electricity consumption and the encrypted data is sent to the network operator. The transmission channel to the network operator is chosen in such a way that the identity of the smart meter remains secret.  This prevents the smart meter from exposing itself during communication with the operator. Groups are formed among smart meters and a smater meter aggregator is selected in each group. The aggregator is selected randomly and all smart meters send their keys to this aggregator. Subsequently, the keys are summed up at the aggregator and sent to the network operator. The network operator receives a single key and with this key it can only decrypt the messages from one smart meter group. additionally, the operator has to add up all the messages and only then it will be possible to decrypt the messages. There is a possibility that aggregator cooperates with the network operator. The aggregator would then be able to send individual keys from smart meters to the operator. While the operator would not be able to match the key to any message, by brute force it could decrypt all messages with that key and see which decrypted message has meaningful content. To prevent this attack, an additional measure is taken. All smart meters in a group organize themselves topologically in a ring structure. In this ring structure, all smart meters cooperate with each other and change their keys every round in such a way that the individual key of a smart meter changes, but not the summed key of all smart meters. Even if the aggregator forwards the keys to the network operator, they would no longer be valid in the next round.(nochmal nachlesen, warum nicht eine individuelle Nachricht entschlüsselt werden kann) A disadvantage of this procedure is that if a smart meter leaves the group, then a new aggregated key must be formed.(Privacy-enhanced architecture for smart metering)\\
\\
\textbf{Battery Solutions}
\\
\\
The battery approach describes a household with a connected battery that is charged, e.g. by grid purchase or photovoltaic panels. The goal of the approach is that the battery feeds energy into the household in such a way that the grid operator can no longer detect private information based on the electricity consumption. \\
\begin{figure}[tbp]
  \centering
  \includegraphics[width=1\textwidth]{images/Battery.jpg}
  \caption[Short description]{The power consumption of a household in a day with battery.}
  \label{fig:Battery}
\end{figure}
The figure shows the electricity consumption of a private household with a connected battery that is charged via solar panels. It can be seen 3 lines. The red line shows the electricity consumption of the household. The green line shows when the battery is discharged(when the battery is feeding power to the household). The blue line is the power consumption that the grid operator can see.In the figure you can see that when the battery brings electricity to the household, then grid operator sees that a household does not consume electricity. In other cases, the grid operator sees that electricity is being consumed, but it is much less than the house actually consumes because the battery offsets some of the electricity consumption. In other words, if a household is connected to a battery, the grid operator cannot make correct statements about the behavior of the people in the household. \\
An algorithm for batteries was proposed in(Protecting consumer privacy
from electric load monitoring). This method uses an algorithm that can control the battery to produce a constant characteristic curve in consumption.  The algorithm targets a static and fixed current consumption. The target consumption is calculated differently by the algorithm depending on the house consumption and battery capacity. If the power consumption is below the consumption set by the algorithm, then the battery is charged with the difference from the target consumption. If the power consumption is above the target consumption set by the algorithm, the battery is discharged with the difference from the target consumption. If the consumption is significantly higher, so that the battery can no longer absorb the additional consumption, then it is switched to recovery mode. In recovery mode, the target consumption is temporarily increased, so that the battery can charge on the side, even though the house is currently consuming a lot of power. If the recovery mode can be switched off, then a new target consumption is calculated based on the new data. It is important to remember that this method does not anonymize power consumption, so it is even more important to measure how much information can still be extracted from power consumption. There are the following metrics to calculate how much privacy is gained by the algorithm.
\begin{enumerate}
\item Relative Entropy:\\
Relative entropy is used to compare two sources of information. In this case it would be the power consumption with the algorithm and the power consumption without the algorithm. These two loads form a stochastic process and can then be compared with the relative entropy. (Affordable privacy for home
smart meters)
\item Cluster Classification:\\
The cluster classification has already been explained for the NILM method and described in this paper at [ref].  The cluster classification has already been explained for the NILM method and described in this paper at [ref]. It can also be used as a metric to evaluate privacy. Here one would perform a cluster analysis with the battery method and once without. Then one looks at the number of clusters in both measurements and if fewer clusters are found with the battery method, then this is considered a privacy gain.
\item Regression Analysis:\\
In the regression analysis, first a cross-correlation and afterwards a simple linear regression is performed. More precisely, both power consumptions are "superimposed" at the point of their maximum cross-correlation. Subsequently, a linear regression is performed and the privacy is evaluated on the basis of the quality of the predictor.
\end{enumerate}

\subsection{Metering for Billing}
In order to fully protect the privacy of a household, metering for billing procedures must also be applied. Otherwise, conclusions about electricity consumption can be drawn from the billing. A simple solution would be to increase the frequency of the billing period. But at the same time it is also in the interest of the customer to buy electricity as cheaply as possible. The customer can be offered better electricity contracts if the billing period is shorter. In addition, it cannot be guaranteed that the customer's privacy is not violated in more complex electricity contracts by other features, even if a higher billing period is used. %In the following paragraph, different methods are explained that solve the problem for metering for billing. 
This master thesis focuses mainly on the metering for operations problem. 
By implementing Trusted Platform Modules (TPM) in German smart meters, the problem is considered to be solved. For completeness, frequently proposed solutions in the scientific community are presented.\\
\\
\textbf{Billing with a Trusted Third Party}
\\
\\
The advantages and disadvantages of a TTP have already been explained in the upper section. The principle is similar to metering for operations. The smart meter sends its measurements to the TTP and the TTP calculates the bill over the time period specified in the electricity contract. The billing is then sent to the electricity provider. On paper, this approach is simple, but important practical questions often remain unanswered. For example, who pays the trusted third party? In this case, the trusted third party provides a service to the electricity provider. However, if the electricity provider pays for the service, then the trusted third party is no longer independent.\\
\\
\textbf{Billing with a Trusted Platform Module}
\\
\\
Billing can also be implemented on the smart meter with a Trusted Platform Module. A TPM is a chip that is installed within the smart meter and thus additional security features can be used on the smart meter. The TPM contains a cryptographic processor that can generate random numbers, generate RSA keys, generate SHA-1 hashes and it has
an encryption-decryption-signature engine. In addition, the TPM can be used to prove that nothing has been tampered with the smart meter afterwards. A secured smart meter can therefore perform correct billings at the customers side and the electricity provider can trust the smart meter. However, the TPM is installed by the electricity provider and it only guarantees the validity of the billing. If the electricity provider decides to send additional sensitive information within the calculations in the TPM, the TPM has no advantage for the end user.(Privacy-Aware Smart Metering)\\
\\
\textbf{Billing Secured via Advanced Cryptography}
\\
\\
Lastly, there is the cryptographic commitment method.
With this approach, no other participant needs to be trusted. A smart meter can use a cryptographic commitment to prove that each bill was calculated correctly. So with cryptographic commitments, billing can be done on the customer side.\\
Lastly, there is the cryptographic commitment method.
With this approach, no other participant needs to be trusted. A smart meter can use a cryptographic commitment to prove that each bill was calculated correctly. So with cryptographic commitments, billing can be done on the customer side. A commitment is a cryptographic application and works as follows. Both sides agree on the same commitment procedure. Then it is possible that one side can generate a c=(x,r) as an obligation. Here x would be the calculation and r would be a random number. If one wants to check the commitment for correctness, then there is an Open(c,x,r) function that returns True(if correct) or False(if incorrect).
Cryptographic commitments are mathematically constructed so that it is easy to compute a c=(x,r), but hard to find an x /= x' with an r' such that open(c,x',r') returns True. In this use case, the following procedure is often used. [pedersen] \\
Commit(x, r) · Commit(y, s) = Commit(x+y, r+s)\\
$Commit(x, r){k}$ = Commit(x · k, r · k)\\
The special feature of the method of [pedersen] is that at the same time non-homomorphic properties are satisfied. Without going into exact technical details, cryptographic commitments work as follows in a smart grid.
The smart meter generates a cryptographic commitment for each measurement. Via a public key infrastructure, the smart meter and the electricity provider receive cryptographic keys. With these keys, the smart meter can sign its commitments and then send them to the electricity provider. The electricity provider checks the commitments for correctness and if all data is correct, the electricity provider sends back a list with electricity prices and the corresponding time stamps. The meter now knows at each point in time how much electricity was consumed and what the price was. By exploiting the homomorphic properties of the procedure, the smart meter can now calculate the electricity prices. The electricity price in this case would be the variable k.So the smart meter creates new cryptographic obligations with the electricity price calculated on the consumption and sends these new obligations to the electricity provider. The electricity provider can verify the correctness by performing the same calculations as the smart meter. If the results with the Open(c,x,r) method return true, then the calculations were performed correctly by the smart meter.
This method is suitable for simple electricity tariffs when only a factor on the electricity consumption needs to be calculated. If an electricity contract is more complex with different conditions such as a higher electricity price if a certain electricity consumption is exceeded, then this approach can no longer be implemented.
\section{Technical guideline TR-03109}
This paragraph will discuss the technical guideline published by the BSI (Federal Office for Information Security). The BSI is the entity of the German federal government that deals with digital security issues and issues recommendations as well as mandatory security guidelines for critical infrastructures. Among other things, technical guidelines are published in which security standards are defined for different IT systems. The technical guideline BSI-TR-03109 defines minimum requirements for the functionality, security and interoperability of smart meters in Germany.  The technical guideline BSI-TR-03109 defines minimum requirements for functionality, security and interoperability that individual components of smart meters in Germany must fulfill. The guideline as a whole consists of 6 different documents, which are shown in Figure 3.1. Based on the guidelines, it is possible to have devices certified by test centers. Unless otherwise described, all information are derived from the technical guideline.\\
\\
\textbf{Stakeholder in the Smart Grid}
%hier ein kleiner text
\\

\item Consumer: \\
The consumer is the person who uses electrical energy, gas, water
or heat. In addition, the consumer is the owner of the measurements processed and stored in the SMGW. In order to interact with the SMGW, the consumer uses a communication device. All necessary data can be retrieved and displayed through it.\\
\item SMGW administrator:\\
A Smart Meter Gateway Administrator (GWA) a trusted entity and each SMGW is assigned a GWA. The GWA handles the configuration, monitoring and control of SMGWs and it is even possible to perform updates of SMGWs via the GWA.\\
\item Authorized external entities:\\
External market participants (EMT) are all other authorized participants in the energy network that can establish a communication connection with the SMGW.These include power grid providers and electricity suppliers. The SMGW ignores all other communication requests that do not come from the GWA or EMTs in order to prevent attacks.\\
\end{enumerate}
There are several other actors such as Controllable Local Systems, service technicians and meters. However, these actors do not play a major role in the protocol that is proposed here.
\begin{figure}[tbp]
  \centering
  \includegraphics[width=1\textwidth]{images/BSI-TR-03109.png}
  \caption[Short description]{An example of a NILM analysis.}
  \label{fig:Appliance_Model}
\end{figure}\\
\\
\textbf{Stakeholder Motives}
\\
\\
It has already been explained in this chapter which participants in the power grid interact with each other. Now in particular it will be discussed which motives the different participants have and which malicious motives can be pursued by the participants or by attackers.  \\
\\
The smart meter attempts to achieve the 3 security objectives of confidentiality, integrity and availability. The 3 security goals are often summarized as CIA. Another important security goal for this work is anonymity. The 4 definitions are essential for the understanding of this work. Therefore, the terms are explained below.
\begin{enumerate}
\item Confidentiality: \\
It is not possible for an unauthorized party to gain information about the content of the data sent.
\item Integrity:\\
It is not possible for an unauthorized party to modify the content of data without data without this being noticed.
\item Availability: \\
It is not possible for an unauthorized party to interfere with the functionality of a service.
\item Anonymity:\\
Verbergen der Identität vor dem Kommunikationspartner???
Vllt wer hat die Daten gesendet, wer hat den Stromverbrauch gesendet?
\end{enumerate}
 
 
All available attacks will aim at bypassing these security targets to get information about the customer. 
\\
\\
\textbf{Customer's Motives}
\\
\\
For the costumer, all the security goals defined above are important. But by far the most important is the security goal of anonymity through the smart meter. Possible attacks on the electricity consumption penetrate deeply into the private sphere of each customer. Therefore, no conclusions may be drawn from the electricity consumption of a customer.\\
On the other hand, unethical customers may try to steal electricity to save energy costs. The smart meter is located in or on the customer's house. An unethical customer could attempt to tamper directly with the smart meter's hardware or software. The attempts could look like this, a Costumer could try to reduce the recorded electricity consumption at the Smart meter or the Smart meter could be manipulated to measure less electricity when electricity prices are high and more electricity when electricity prices are low.
\\
\\
\textbf{Electricity Provider's Motives}
\\
\\
For the electricity provider, the authenticity of the billing is the most important security objective because From the electricity provider's point of view, the customer is not trustworthy in the calculation of the bill. In addition, the customer has access to the smart meter at almost any time in an environment trusted by the customer. Unlike analog meters, smart meters cannot be mechanically attacked. But if a customer manages to change the software of the smart meter, the billing can be manipulated at the same time. On the other hand, the electricity provider can also be an overly intrusive electricity provider. In the paragraph [Ref NILM] it was explained how a behavioral analysis can be created from the electricity consumption. This sensitive information could be used to gain an additional source of income. In [SSRN] it was listed which questions could be answered by a Nilm analysis. Quote: "On what days and during what times do you watch TV? How much home time do you spend in front of your computer?" or "Are any of the appliances in your household failing or operating below optimal efficiency? Do you own (and so presumably like) lots of gadgets?". Advertising companies would certainly pay money for this kind of information in order to be able to advertise more accurately. For this reason, it is presumed that the electricity provider is considered an honest-but-curious adversary. 
\subsection{Interfaces and functions of the Smart Meter Gateway}
\begin{figure}[tbp]
  \centering
  \includegraphics[width=0.8\textwidth]{images/Smart_Meter_Interfaces.png}
  \caption[Short description]{An example of a NILM analysis.}
  \label{fig:Appliance_Model}
\end{figure}
A smart meter or as described in the technical guideline a smart meter gateway(SMGW) must provide 3 different physical interfaces.
\begin{enumerate}
\item Local Metrological Network (LMN):\\
The LMN is the communication interface in which communication takes place with the connected meters for energy and material quantities (electricity, gas). An SMGW can communicate with one meter from one end user or with several meters from different end users. In practice, however, one SMGW is often responsible for one meter. The measured values are sent from the meters via the LMN to the SMGW and stored there.
\item Wide Area Network (WAN):\\
The WAN is the only communication interface with which the SMGW can communicate with EMTs or GWAs over the Internet. If a request is made to the SMGW that was not sent by these authorized participants, then the request is discarded and ignored.
\item Home Area Network (HAN):\\
In HAN, an SMGW interacts with Controllable Local Systems (e.g., photovoltaic systems). In addition, users and service technicians can use the HAN interface to display information about power consumption through functions offered by the SMGW.
\end{enumerate}
\begin{figure}[tbp]
  \centering
  \includegraphics[width=1\textwidth]{images/Messwertverarbeitung.png}
  \caption[Short description]{An example of a NILM analysis.}
  \label{fig:Appliance_Model}
\end{figure}
\textbf{Functionality of the smart meter gateway}
\\
\\
First, the task of SMGW is to store the measurements sent by meters from the LMN. Then, the readings are processed in the SMGW and sent to the authorized EMTs in the WAN after processing. An SMGW must also perform the tasks of a firewall and separate the 3 interfaces. It is therefore impossible for an EMT or GWA to make requests to devices located in the HAN or LMN, even if it is allowed to interact with the SMGW over the WAN.
Since the WAN interface is the most important interface for this work, it will be discussed in more detail.\\
\\
\textbf{Functions of the SMGW in the WAN}
\\
\\
The tasks performed by the WAN have already been explained in the paragraph above. Now the functions and security mechanisms offered by the SMGW to guarantee secure interaction on the WAN will be described.
\begin{enumerate}
\item Transmission of measured values based on evaluation and WAN communication profiles:\\
Communication profiles of GWAs are stored in SMGW. The communication profiles determine how the data is processed in the SMGW and forwarded to EMTs.

\item Pseudonymization:\\
Data that is not relevant for billing must be pseudonomized for data protection reasons. For this purpose, the unique identification number that each SMGW has is replaced by a pseudonym. Subsequently, the information is not sent directly to an EMT, but is forwarded to the EMT via the GWA. This additionally protects the identity of the sending SMGW.%vllt als extra part machen. was unten kommt.
Even if peusodonymization does not allow an SMGW to be directly assigned, the described attack in [ref] and the resulting behavioral analysis is still possible. Since no other security mechanisms are available from the SMGW, the question must be asked whether pseudonymization as proposed in the technical guideline is sufficient.

\item Time synchronization:\\
In order for the cost electricity consumption to be calculated correctly, it is essential that the SMGW have an accurate time. For this purpose, the system time of the SMGW is synchronized with the time server of the GWA at regular intervals.

\item Wake-Up Service:\\
A GWA is able to force a communication link with the SMGW. This is done via a data packet signed by the GWA. The SMGW then establishes a fixed preconfigured communication connection to the GWA. This enables the GWA to execute administration commands on the SGMW. 
\end{enumerate}
\subsection{Eavesdropping}
Eavesdropping may be the weakest type of attack, but successful eavesdropping on the communications of the smart meter could be useful to e.g. intruders. However, curious neighbors might also have an interest in the behavior inside the house. Turning on/off lights implies that someone is at home or leaving the house. Therefore, eavesdropping on electricity consumption could provide information about when is a suitable time to break in. To prevent eavesdropping, smart meter communication is encrypted to maintain confidentiality. Cryptographic algorithms such as AES are widely used today and have been analyzed for weaknesses over the years by a number of researchers. Hence, a successful attack on encrypted data to extract information is extremely unlikely.
\subsection{Active Attackers}
In Germany, the smart grid is one of the critical infrastructures. This means that the failure of the smart grid could lead to a significant compromise of public safety or other serious consequences. Such systems are threatened by active attackers. The objective of active attackers may not necessarily be to analyze a user's electricity consumption. They may want to disable availability through e.g. denial of service attacks. These attacks could leave major damage to the power grid and are definitely a realistic threat [quelle]. But this thesis focuses on smart meters and the anonymization of electricity consumption. That's why it is assumed that the active attacker does not carry out system-wide attacks on the power grid. Rather, it is assumed that the attacker attempts to take control in the proposed DC network. Among other things, it is assumed that the attacker has the theoretical ability to take over one or more SMGW and send messages through the SMGW. In addition, if the attacker has taken over an SMGW, it can perform all operations that are possible through the proposed DC network. In the next section, the conceptual solution of the DC-Net is proposed and how the DC-Net could be implemented in the technical guideline of the BSI. It also describes which attacks on the DC-Net are possible with the defined attacker model.

\cleardoublepage

%%% Local Variables:
%%% TeX-master: "diplom"
%%% End:
