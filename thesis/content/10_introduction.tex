\chapter{Introduction}
\label{sec:intro}

% Die Einleitung schreibt man zuletzt, wenn die Arbeit im Großen und
% Ganzen schon fertig ist. (Wenn man mit der Einleitung beginnt - ein
% häufiger Fehler - braucht man viel länger und wirft sie später doch
% wieder weg). Sie hat als wesentliche Aufgabe, den Kontext für die
% unterschiedlichen Klassen von Lesern herzustellen. Man muß hier die
% Leser für sich gewinnen. Das Problem, mit dem sich die Arbeit befaßt,
% sollte am Ende wenigsten in Grundzügen klar sein und dem Leser
% interessant erscheinen. Das Kapitel schließt mit einer Übersicht über
% den Rest der Arbeit. Meist braucht man mindestens 4 Seiten dafür, mehr
% als 10 Seiten liest keiner.

%Indeed, the fact that it is difficult to grasp intuitively the potentially far-reaching consequences of seemingly benign data collection practices makes the process of developing regulations in the absence of a formal framework all the more difficult.

%Aggregation is beneficial to privacy, but examples have also shown that this is not sufficient to a priori rule out the possibility of significant privacy breaches

%re-identification attacks

%firms might want to interact with each other or meet public reporting requirements while trying to hide proprietary information. A concrete example of such a situation arises in electricity markets,where inverse optimization techniques can be used to identify the cost functions of energy producers, using only the information published about daily market outcomes setting the electricity prices (Ruiz et al. 2013).

%linkage attacks

%since we obviously do not want to rule out such analyses, what differential privacy aims for is not to prevent information disclosure per se, but to guarantee that if an individual provides her data, it does not become significantly easier to make new inferences about that specific individual compared to the situation where her records is not in the dataset.

%statistical disclosure limitation

\section{A Section}


\section{Another Section}

\cleardoublepage
%Additionally the combined consumption of all users makes in the first place real-pricing possible. This is instrumental for grid provider to stimulate customer using or not using their electrical device in for the grid provider convenient times. In the future will be this feature more and more necessary to avoid peaks. Through the climate crisis triggered energy transition is it highly likely that for example electrical vehicles and heat pumps are more and more components of private households. Offering the incentives to charge the car or preheating watertanks in more favorable times is only possible with a time-based pricing.
%%% Local Variables:
%%% TeX-master: "diplom"
%%% End:
