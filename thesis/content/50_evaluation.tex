\chapter{Experiments}
\label{sec:evaluation}

% Zu jeder Arbeit in unserem Bereich gehört eine Leistungsbewertung. Aus
% diesem Kapitel sollte hervorgehen, welche Methoden angewandt worden,
% die Leistungsfähigkeit zu bewerten und welche Ergabnisse dabei erzielt
% wurden. Wichtig ist es, dem Leser nicht nur ein paar Zahlen
% hinzustellen, sondern auch eine Diskussion der Ergebnisse
% vorzunehmen. Es wird empfohlen zunächst die eigenen Erwartungen
% bezüglich der Ergebnisse zu erläutern und anschließend eventuell
% festgestellte Abweichungen zu erklären.


%statistical analysis and signal processing are being increasingly applied to private data obtained from individuals
The DC network from chapter 3 ensures anonymity of user data in a smart grid. It is impossible for an electricity provider to obtain the individual electricity consumption of a user without performing aggressive attacks. But this criterion alone is insufficient. In chapter two, it was described that 52 different household appliances could be correctly identified from the aggregated electricity consumption of two houses with a accuracy of 55 percent a measurement interval of 1 hour. Accordingly, it is still possible to draw conclusions about individual electricity consumption or about the use of household appliances from different households from an aggregated result. In addition, the measurement interval in the German smart grid is much more frequent and amounts to 15 minutes. To make an analysis over the aggregated result impossible, the DC network must have a minimum number of participants. For this purpose, an experimental analysis was performed in the master thesis to be able to make a statement about the minimum number of participants in a DC network. When researching the information content of aggregated data, only sparse information and no common metric could be found. In particular, of interest is: How much electricity consumption from individual households needs to be aggregated so that certain characteristics, such as the use of individual household appliances, can no longer be analyzed from the results? In this chapter, the experiment investigates from how many participants in the DC network it can be said with certainty that it is no longer possible to draw conclusions about individual users from the result. For this purpose, the assumption and the structure of the experiment and its execution will be described first. Afterwards the results of the experiment are considered and analyzed.
\\
\\
\textbf{Assumption}
\\
\\
The goal of the experiment is that no inferences can be drawn from the result of the aggregated electricity consumption. This includes that no persons can be deanonymized from the result or that no household appliances can be identified from the result. There is no metric that can measure anonymity from electricity consumption, but there are metrics that can measure the information content of electricity consumption. The assumption is therefore that if the information content from the aggregated result of electricity consumption is low, then no conclusions can be drawn about households or household appliances.
\\
\\
\textbf{Experimental Setup}
\\
\\
Ref[relative entropy] described which metrics can calculate the information content of aggregated electricity consumptions. For the experiment, a different procedure was chosen. Electricity consumptions from the London Smart Meter dataset were used. In the dataset, the electricity consumption of 5,567 different households between 2011 and 2014 was recorded with measurement intervals of 30 minutes each. In the experiment, n houses were selected and their electricity consumptions were aggregated for each time point. From the aggregated result, the standard deviation for each individual timestamp was calculated and compared to the individual electricity consumption of a house at that time. The comparison looks if the electricity consumption of a single house is within the tolerance limit of the aggregated result. The tolerance limit is the following interval (aggregated erg- standard deviation<standard deviation<aggregated erg+ standard deviation).  If the electricity consumption of a house is above or below the tolerance limit at a point in time, then it is assumed that the electricity consumption of the house has a high information content at that point in time and that this information content could be read from the result. For each timestamp, each house is counted individually for how often the individual power consumptions are outside the tolerance limit. In order to perform the experiment, a script was programmed in R, which selects random houses from the London Smart Meter data set. Additionally, we filtered for houses that have readings between 2013 and 2014. For the highest possible precision of the calculation, it was ensured that when a comparison calculation is performed, the timestamp is also correctly identical. In short every 30 minutes measurements in 2013-2014 of a house were compared whether the electricity consumption is above or below the tolerance limit of the aggregated results of N houses. For each calculation, the number of times the values are outside the tolerance limit is counted. This experiment was repeated with different N sizes to analyze how strong the effects of the information content in the aggregated result are measurable.
\\
\\
\textbf{Results}
\\
\\
Figure 5.1 - 5.8 show the results of the experimental tests. Experiments were conducted with 2, 3, 5, 10, 25, 50, 100 and 150 houses, whose electricity consumptions were aggregated. On the X-axis the time in months can be read. On the Y-axis, the average daily electricity consumption within half an hour of a house can be read. If the figure is considered, it can be seen that in the experiments with little aggregated houses (2,3,5,10) more significant variations in the amplitude of the characteristic curves are visible. This can be explained by the fact that in the case of less aggregated houses, individual power consumptions have a greater impact and can therefore have a more significant effect on the overall result. This could be an indication that the information content of a single house can still be determined. From 25 - 150 houses the characteristic curves approach the same pattern. The seasons can be clearly recognized. This is directly evident in the house groups 2-10. The fact that the seasons can be recognized so clearly, can be argued indirectly with the fact that a larger house group is observed. This is because individual household appliances no longer stand out clearly and larger patterns of behavior can be recognized, such as the fact that in the winter, people spend more time inside their homes, while in the summer, they spend more time outside. In addition, it is noticeable that the basic load in each diagram is different, however, quickly converges. The base load can best be seen in the boxplot diagram. The boxplot diagram almost diagram 5.1 -5.8 together as boxplot. In conducting the experiments, the preliminary assumption was that with larger aggregations, the result would approach a horizontal constant. The approximation of a constant would be clear evidence that the information content in the result was lost. The experiment cannot prove the prior conjecture. If the results were to approach a constant at larger aggregations, then the upper quartile and the lower quartile would become narrower piece by piece. However, this observation cannot be gleaned from the boxplot.

%Nachschauen mit den ferien im april bei 2 häusern
%durchschnittliche Mittelwert(von boxplots)
%ergebnisse der abweichung
%versuche wurden 2 mal wiederholt.
\afterpage{%
\begin{figure}[!p]
\centering
\includegraphics[width=0.85\textwidth]{images/Aggregated Electricity Consumption of 2 Houses.png}
\caption[Bandbreite Diagramm mit aktivierter Verschlüsselung]{}
\label{img:bytes-e}
\end{figure}
\begin{figure}[!p]
\centering
\includegraphics[width=0.85\textwidth]{images/Aggregated Electricity Consumption of 3 Houses.png}
\caption[Diagramm der Bandbreite ohne Verschlüsselung]{}
\label{img:bytes-we}
\end{figure}
\clearpage
}
\afterpage{%
\begin{figure}[!p]
\centering
\includegraphics[width=0.85\textwidth]{images/Aggregated Electricity Consumption of 5 Houses.png}
\caption[Bandbreite Diagramm mit aktivierter Verschlüsselung]{}
\label{img:bytes-e}
\end{figure}
\begin{figure}[!p]
\centering
\includegraphics[width=0.85\textwidth]{images/Aggregated Electricity Consumption of 10 Houses.png}
\caption[Diagramm der Bandbreite ohne Verschlüsselung]{}
\label{img:bytes-we}
\end{figure}
\clearpage
}
\afterpage{%
\begin{figure}[!p]
\centering
\includegraphics[width=0.85\textwidth]{images/Aggregated Electricity Consumption of 25 Houses.png}
\caption[Bandbreite Diagramm mit aktivierter Verschlüsselung]{}
\label{img:bytes-e}
\end{figure}
\begin{figure}[!p]
\centering
\includegraphics[width=0.85\textwidth]{images/Aggregated Electricity Consumption of 50 Houses.png}
\caption[Diagramm der Bandbreite ohne Verschlüsselung]{}
\label{img:bytes-we}
\end{figure}
\clearpage
}
\afterpage{%
\begin{figure}[!p]
\centering
\includegraphics[width=0.85\textwidth]{images/Aggregated Electricity Consumption of 100 Houses.png}
\caption[Bandbreite Diagramm mit aktivierter Verschlüsselung]{}
\label{img:bytes-e}
\end{figure}
\begin{figure}[!p]
\centering
\includegraphics[width=0.85\textwidth]{images/Aggregated Electricity Consumption of 150 Houses.png}
\caption[Diagramm der Bandbreite ohne Verschlüsselung]{}
\label{img:bytes-we}
\end{figure}
\clearpage
}
\afterpage{%
\begin{figure}[!p]
\centering
\includegraphics[width=0.85\textwidth]{images/ggplotTest.png}
\caption[Bandbreite Diagramm mit aktivierter Verschlüsselung]{}
\label{img:bytes-e}
\end{figure}
\begin{figure}[!p]
\centering
\includegraphics[width=0.85\textwidth]{images/test.png}
\caption[Diagramm der Bandbreite ohne Verschlüsselung]{}
\label{img:bytes-we}
\end{figure}
\clearpage
}

\ldots Experiments \ldots

\todo{write evaluation}

\cleardoublepage

%%% Local Variables:
%%% TeX-master: "diplom"
%%% End:
