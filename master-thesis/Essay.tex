%%
%% This is file `sample-sigconf.tex',
%% generated with the docstrip utility.
%%
%% The original source files were:
%%
%% samples.dtx  (with options: `sigconf')
%% 
%% IMPORTANT NOTICE:
%% 
%% For the copyright see the source file.
%% 
%% Any modified versions of this file must be renamed
%% with new filenames distinct from sample-sigconf.tex.
%% 
%% For distribution of the original source see the terms
%% for copying and modification in the file samples.dtx.
%% 
%% This generated file may be distributed as long as the
%% original source files, as listed above, are part of the
%% same distribution. (The sources need not necessarily be
%% in the same archive or directory.)
%%
%% The first command in your LaTeX source must be the \documentclass command.

\documentclass[english,sigconf,11pt]{acmart}
\usepackage[english]{babel}
\usepackage{lmodern}
\usepackage {enumitem}
\usepackage[section]{placeins}
\usepackage{graphicx}
\usepackage{caption}


%%
%% \BibTeX command to typeset BibTeX logo in the docs
\AtBeginDocument{%
  \providecommand\BibTeX{{%
    \normalfont B\kern-0.5em{\scshape i\kern-0.25em b}\kern-0.8em\TeX}}}

\settopmatter{printacmref=false}
\renewcommand\footnotetextcopyrightpermission[1]{}
\pagestyle{plain}
%% Rights management information.  This information is sent to you
%% when you complete the rights form.  These commands have SAMPLE
%% values in them; it is your responsibility as an author to replace
%% the commands and values with those provided to you when you
%% complete the rights form.
%\setcopyright{acmcopyright}
%\copyrightyear{2018}
%\acmYear{2018}
%\acmDOI{10.1145/1122445.1122456}

%% These commands are for a PROCEEDINGS abstract or paper.
%\acmConference[Woodstock '18]{Woodstock '18: ACM Symposium on Neural
%  Gaze Detection}{June 03--05, 2018}{Woodstock, NY}
%\acmBooktitle{Woodstock '18: ACM Symposium on Neural Gaze Detection,
%  June 03--05, 2018, Woodstock, NY}
%\acmPrice{15.00}
%\acmISBN{978-1-4503-XXXX-X/18/06}
\usepackage{array}
%\newcolumntype{P}[1]{>{\centering\arraybackslash}p{#1}}
\newcolumntype{M}[1]{>{\centering\arraybackslash}m{#1}}

%%
%% Submission ID.
%% Use this when submitting an article to a sponsored event. You'll
%% receive a unique submission ID from the organizers
%% of the event, and this ID should be used as the parameter to this command.
%%\acmSubmissionID{123-A56-BU3}

%%
%% The majority of ACM publications use numbered citations and
%% references.  The command \citestyle{authoryear} switches to the
%% "author year" style.
%%
%% If you are preparing content for an event
%% sponsored by ACM SIGGRAPH, you must use the "author year" style of
%% citations and references.
%% Uncommenting
%% the next command will enable that style.
%%\citestyle{acmauthoryear}
 
%%
%% end of the preamble, start of the body of the document source.
\begin{document}
\selectlanguage{english}
%%
%% The "title" command has an optional parameter,
%% allowing the author to define a "short title" to be used in page headers.
\title{Essay}

%%
%% The "author" command and its associated commands are used to define
%% the authors and their affiliations.
%% Of note is the shared affiliation of the first two authors, and the
%% "authornote" and "authornotemark" commands
%% used to denote shared contribution to the research.
\author{Gregor Garten}
\email{gregor.garten@mailbox.tu-dresden.de}
\affiliation{%
  \institution{Technische Universit\"at Dresden}
  \city{Dresden}
  \state{Sachsen}
}


%%
%% By default, the full list of authors will be used in the page
%% headers. Often, this list is too long, and will overlap
%% other information printed in the page headers. This command allows
%% the author to define a more concise list
%% of authors' names for this purpose.

%%
%% The abstract is a short summary of the work to be presented in the
%% article.
%%\begin{abstract}
%%Internet traffic has grown in the last years because of the rising amount of mobile users \cite{statista}. Due to new services complexity in terms of data volume and frequency of requests are increasing heavily especially for large streaming provider like Netflix or Google. Even the widely-used architectural design REST, which is known for scalability, comes to a limit at such a large scale. Since on mobile devices performance issues are quickly noticeable, low latency and efficient data fetching becomes necessary.
%%This is the reason why Netflix published a own framework called Falcor to tackle the problems of data fetching. In this paper an overview of REST and Falcor are presented and to compare both approaches an experiment in a real world environment was executed. At the end the performance regarding latency and data volume is analyzed showing which approach is more suitable for data fetching.
%%\end{abstract}

%%
%% The code below is generated by the tool at http://dl.acm.org/ccs.cfm.
%% Please copy and paste the code instead of the example below.
%%

%%
%% Keywords. The author(s) should pick words that accurately describe
%% the work being presented. Separate the keywords with commas.
%\keywords{Falcor, Latency, Web Services, Framework, Data fetching}

%% A "teaser" image appears between the author and affiliation
%% information and the body of the document, and typically spans the
%% page.


%%
%% This command processes the author and affiliation and title
%% information and builds the first part of the formatted document.
\def\@copyrightspace{\relax}

\maketitle

\section{Problem statement}
The electrical grid is facing a structural transformation through the increase of power sources. Instead of few power plants, which are producing a high amount of electricity, the electrical grid is changing to more and more smaller producers like wind turbines or other renewable technologies. Even single-family houses are able to inject electricity into the power grid through solar panels installed on the roof. As a consequence the electrical grid gets a third participant the so called prosumers who can consume electricity as well as produce it. The new distribution of consumers, producers and prosumers poses a difficult task for grid operators since it becomes harder to track electricity supply and demand. To tackle this issue, smart meters are introduced to the power grid by installing them in the households. Smart meters can send the current electricity consumption of the customers to the electricity provider solving the problem of demand and supply. However, that creates a new privacy problem for the customers. Smart meters usally communicate every 15 minutes with the electricity provider enabling the possibility for a accurate behavior analysis of the customer like daily routines and religious affiliation. In this master thesis the goal is to implement a solution which provides all the advantages of smart meters like better electricity rate, but also solve the privacy issues.
\section{Motivation}
The introduction of smart meters into the power grid poses a new challenge to the privacy of households. First, the higher resolution of data allows a behavior analysis based on electricity usage patterns. This information could be stored by electrical providers and used further for specialized advertisement on households. Second, some individuals may want to hide their electricity consumption now or in the future. Because global warming steadily raises the awareness of the society and sustainable living becomes more important. Therefore a high electricity consumption can harm the image of people \cite{spiegel}.

%3A nachschauen seite 17
\section{Background}
%To achieve real privacy with smart meter it is necessary that 2 sub-problems
The smart meter privacy problem can be divided into two sub-problems: Metering for Billing and Metering for Operation. Only if those two problems are handled properly it is possible to achieve real privacy. In metering for operation the goal is to implement a privacy preserving solution which prevents a behavior analysis based on electricity usage patterns. Since the electrical provider only needs an overview of the electrical consumption of all smart meters in an area.
Furthermore it is important to find a solution for metering for billing that makes it impossible to indirectly learn about the electricity consumption by billing the customer. At the same time the billing has to be valid and cant be tampered. This master thesis will primary focus on metering for operation.\\
First approaches of the analysis of electricity usage patterns can be found in 1999 \cite{795138}.
Afterwards several processes are suggested in different works to solve the issue. A common approach is to hire a trusted third party, which is another agent in the power grid trusted by all participants. The trusted third party act as a computation center, which receives all information from  smart meters in a specific location, computes it and sends the anonymized information to the electricity provider. Therefore the electricity provider can't assign the data to specific households.
Another feasible solution is a implementation of a homomorphic cryptographic scheme. These are asymmetric algorithms with the possibility of basic operations like addition on the ciphertext. That allows the smart meter to encrypt its information and send it to a central storage. In the central storage the encrypted messages of all smart meters are summed together and a electricity provider can ask for that information to gain the total electricity consumption of a region. More approaches can be found in \citep{fan2012smart}\citep{finster2015privacy}.

\section{Our approach and Evaluation}
This master thesis will focus on a solution which is called dining cryptographer network often referred as DC-net. It is invented by David Chaum and secures anonymous communication. 
In a previous work two small power grids are created with Gridlab-D. The task is to implement a DC-net into the simulator and analyse which effects the DC-net has on the simulations and if sender anonymity is achievable for the smart meters. If smart meters can communicate anonymously with the electricity provider, the provider can't assign the data to specific households.
Afterwards different questions need to be answered in a evaluation. For example how much information can be extracted out of the aggregated data from the smart meters or how much smart meters have to be aggregated to attain a specific level of privacy.
Therefore the collected data has to be visualized to gain an overview and to  simplify the analysis of the results.
Also a interesting approach is to create an experimental environment which emulates a micro power grid. For this various raspberrypis would imitate real components of a power grid, which would communicate with the proposed solution. The insights of the experimental environment could additionally prove the results of the simulation or it could be a first hint for refutation of the approach. 
%Therefore several measurements are of interest.  
%In a DC-net there are rounds and in every round a participant has to send a message to all other participants. If a participant has no meaningful message to send or doesn't want to send a message, he can transmit a empty message. But previously exchanged keys are xored on the messages.
%Otherwise it would be possible to read private messages, if for example every one is sending an empty message except one participant.
%every message is xored with exchanged keys of participants of the net. 
\section{Solving the problem}
%Implementing a DC-net into a power grid simulation should lead to sender anonymity for smart meters. 
Smart meters should achieve sender anonymity if a DC-net is implemented into the power grid simulation. Therefore smart meters could communicate the aggregated overall electricity consumption without fearing, that the electrical provider is misusing the private data of individual households as it is not feasible to gain information of the attained data.
%EInfach sagen, dass mit einem dc netz die eigenschaft der senderanonymität erreicht wird. wenn man diese eigenschaft in einem power grid hat, dann kann ein smart meter problemlos mit dem electricity provider kommunizieren ohne befürchten zu müssen, dass die daten ausgespäht/missbraucht werden.


\section{Timeplan}

\begin{table}[!htb]
\scriptsize
\centering
\begin{tabular}{|M{3,6cm}|M{3,6cm}|}
\hline 
Week & Task  \\ 
\hline 
Week 1 - 4 & review literature and common proposed solutions for related work and state of the art chapter.    \\ 
\hline 
Week 5 & Start design phase for an implementation into Gridlab-D. Find potential parts of the software where the algorithm can be implemented  without restrictions for further extensions (if something needs to be fixed/changed later)   \\ 
\hline 
Week 6 - 7  & Implementation of the DC-Net into Gridlab-D   \\ 
\hline 
Week 8 - 9 & Setting up an experimental environment with raspberrypis for additional collection of results\\ 
\hline 
Week 10 - 12 & Run different experiments. Collect results. Maybe corrections on the algorithm/experimental setup if problems occur. \\ 
\hline 
Week 13 - 15 & Analyse results. Create graphs to illustrate the results. Start writing. \\ 
\hline 
Week 16 - 20 & Continue writing. Start correction phase of the master thesis. \\ 
\hline 
Week 21 & End correction phase of master thesis. \\ 
\hline 
Week 22 & Some time for final details which are left over during the previous weeks. \\ 
\hline 
\end{tabular} 
\captionof{table}{Timeplan for the master thesis.} 
\normalsize
\end{table}
 
%%Phones, Tablets and other portable devices became highly popular in the recent years, reaching a point where they can not be unimagined in a today's society.
%%Caused by the increasing popularity the amount of internet traffic induced by mobile devices has reached almost 50 \% in Germany \cite{sz-quelle}. Therefore inadequate performance, when fetching data, can lead to a visible loss of quality. This can be explained because of a higher overhead in wireless protocols and due to a longer network round trip time.
%%Hence, managing the growing data portion becomes a challenge for service providers\cite{cederlund2016performance}.
\\
%%To overcome most of the unnecessary traffic, server-side logic like rendering was steadily outsourced into the client-side over the past years. Furthermore an architectural style, which is called representational State Transfer (REST), is commonly used for communication between server and client. REST reduces the latency and the load but also provides options for scalability. But REST comes also with limitations and strict coding guide lines. Those limitations can lead to implementation difficulties for APIs or worse the overall performance. In 2015 Netflix released a framework named Falcor to approach the problems coming with REST. 
%\begin{figure}
%  \includegraphics[width=\linewidth]{network-diagram.png}
%  \caption{Client Server Model of Falcor}
%\end{figure}
%One of Falcor's characteristics is the connection of every data source used in an application server and the representation of the sources in an single JSON Graph, which is communicating in an asynchronous way with the user. That means if an user requests data from the application, Falcor generates a query to the application server getting the data on demand. Afterwards the data is rendered on client-side to a view, which can be extended through several asynchronous queries. A prior advantage to this approach is that different data sources, which are often needed to create a single view, can requested  simultaneously in one query. Another advantage is, that asked data is stored in-memory.\\ Therefore internet traffic should be saved additionally, because data is already cashed and there is no need to produce another query. Figure 1 shows a Falcor Model with client server communication and different data sources located in the cloud. 



%Falcors main goal is to reduce the latency even more.


%Mobile devices belong to the most used internet devices in the recent years. Hence, an increasing amount of the internet traffic is induced by mobile 










\bibliography{essay}
\bibliographystyle{plain}                                
\addcontentsline{toc}{chapter}{Literaturverzeichnis}  








\end{document}
\endinput
%%
%% End of file `sample-sigconf.tex'.
