\chapter{Introduction}
\label{sec:intro}

% Die Einleitung schreibt man zuletzt, wenn die Arbeit im Großen und
% Ganzen schon fertig ist. (Wenn man mit der Einleitung beginnt - ein
% häufiger Fehler - braucht man viel länger und wirft sie später doch
% wieder weg). Sie hat als wesentliche Aufgabe, den Kontext für die
% unterschiedlichen Klassen von Lesern herzustellen. Man muß hier die
% Leser für sich gewinnen. Das Problem, mit dem sich die Arbeit befaßt,
% sollte am Ende wenigsten in Grundzügen klar sein und dem Leser
% interessant erscheinen. Das Kapitel schließt mit einer Übersicht über
% den Rest der Arbeit. Meist braucht man mindestens 4 Seiten dafür, mehr
% als 10 Seiten liest keiner.

%Indeed, the fact that it is difficult to grasp intuitively the potentially far-reaching consequences of seemingly benign data collection practices makes the process of developing regulations in the absence of a formal framework all the more difficult.

%Aggregation is beneficial to privacy, but examples have also shown that this is not sufficient to a priori rule out the possibility of significant privacy breaches

%re-identification attacks

%firms might want to interact with each other or meet public reporting requirements while trying to hide proprietary information. A concrete example of such a situation arises in electricity markets,where inverse optimization techniques can be used to identify the cost functions of energy producers, using only the information published about daily market outcomes setting the electricity prices (Ruiz et al. 2013).

%linkage attacks

%since we obviously do not want to rule out such analyses, what differential privacy aims for is not to prevent information disclosure per se, but to guarantee that if an individual provides her data, it does not become significantly easier to make new inferences about that specific individual compared to the situation where her records is not in the dataset.

%statistical disclosure limitation
\todo{adopt title page}

\todo{adopt disclaimer}

\todo{write introduction}

\section{A Section}

Referencing other chapters: \ref{sec:state} \ref{sec:design}
\ref{sec:implementation} \ref{sec:evaluation} \ref{sec:futurework}
\ref{sec:conclusion}

\begin{table}[htp]
  \centering
  \begin{tabular}{lrr}
    \textbf{Name} & \textbf{Y} & \textbf{Z} \\
    \hline
    \textit{Foo} & 20,614 & \SI{23}{\percent} \\
    \textit{Bar} & 9,914 & \SI{11}{\percent} \\
    \textit{Foo + Bar} & 30,528 & \SI{34}{\percent} \\
    \hline
    \textit{total} & 88,215 & \SI{100}{\percent} \\

  \end{tabular}
  \caption[Some interesting numbers]{Various very important looking numbers and sums.}
  \label{tab:numbers}
\end{table}

More text referencing Table~\ref{tab:numbers}.

\section{Another Section}

\begin{figure}[tbp]
  \centering
  \includegraphics[width=0.8\textwidth]{images/squirrel}
  \caption[Short description]{A long description of this squirrel figure.
  Image taken from
  \url{http://commons.wikimedia.org/wiki/File:Sciurus-vulgaris_hernandeangelis_stockholm_2008-06-04.jpg}}
  \label{fig:squirrel}
\end{figure}

Citing \cite{bellard2005qfa} other documents \cite{bellard2005qfa, boileau06}
and Figure~\ref{fig:squirrel}.

Something with umlauts and a year/month date:
\cite{becher04:_feurig_hacken_mit_firew}.

And some online resources: \cite{green04}, \cite{patent:4819234}

\section{Yet Another Section}

\todo{add content}

\begin{figure}[tbp]
 \missingfigure{Come up with a mindblowing figure.}
 \caption{A mindblowing figure}
 \label{fig:todo}
\end{figure}

\section{Test commands}

\drops \LLinux \NOVA \QEMU
\texttt{memcpy}
A sentence about BASIC. And a correctly formatted one about ECC\@.

\cleardoublepage

%%% Local Variables:
%%% TeX-master: "diplom"
%%% End:
