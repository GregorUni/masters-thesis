\chapter{Technical Background}
\label{sec:state}

% Hier werden zwei wesentliche Aufgaben erledigt:

% 1. Der Leser muß alles beigebracht bekommen, was er zum Verständnis
% der späteren Kapitel braucht. Insbesondere sind in unserem Fach die
% Systemvoraussetzungen zu klären, die man später benutzt. Zulässig ist
% auch, daß man hier auf Tutorials oder Ähnliches verweist, die hier auf
% dem Netz zugänglich sind.

% 2. Es muß klar werden, was anderswo zu diesem Problem gearbeitet
% wird. Insbesondere sollen natürlich die Lücken der anderen klar
% werden. Warum ist die eigene Arbeit, der eigene Ansatz wichtig, um
% hier den Stand der Technik weiterzubringen? Dieses Kapitel wird von
% vielen Lesern übergangen (nicht aber vom Gutachter ;-), auch später
% bei Veröffentlichungen ist "Related Work" eine wichtige Sache.

% Viele Leser stellen dann später fest, daß sie einige der Grundlagen
% doch brauchen und blättern zurück. Deshalb ist es gut,
% Rückwärtsverweise in späteren Kapiteln zu haben, und zwar so, daß man
% die Abschnitte, auf die verwiesen wird, auch für sich lesen
% kann. Diese Kapitel kann relativ lang werden, je größer der Kontext
% der Arbeit, desto länger. Es lohnt sich auch! Den Text kann man unter
% Umständen wiederverwenden, indem man ihn als "Tutorial" zu einem
% Gebiet auch dem Netz zugänglich macht.

% Dadurch gewinnt man manchmal wertvolle Hinweise von Kollegen. Dieses
% Kapitel wird in der Regel zuerst geschrieben und ist das Einfachste
% (oder das Schwerste weil erste).

% Background Warum ist das Thema wichtig?
% Was ist das Smart Grid -> mit Background verknüpfen
% Vor, Nachteile vom Smart Grid?
% Was sind Smart Meter
% Warum sind Smart Meter Privacy relevant?
% DC-Netze einführen
% Related Work: Welche andere Verfahren gibt es?
% Vllt erklärung regeln vom BSI

This section introduces an overview of the basic concepts for this work. Therefore, the key components of the smart grid are explained, what structural changes and what challenges the smart grid will bring.
In addition, this chapter discusses the current state of research.\\

\section{Smart Grid}
The original energy network was mainly considered as a transmission system to send electricity from the generators via a elongated network of cables and transformers to the consumers.% vllt hier noch schreiben, dass es erneuerbare Energien wegen der Klimakriese gibt
Instead of a few electricity producers (e.g. nuclear power plants, coal-fired power plants), which were responsible for a large part of the electricity generation, there are now many smaller producers (e.g. wind turbines). %However, renewable power generation is often dependent on external environmental influences. Therefore, smart meters have been widely introduced in households to ensure that the smart grid is stable despite fluctuations in power generation.
However, renewable power generation is often dependent on external environmental factors. In order for the smart grid to be stable despite fluctuations in power generation, smart meters have been introduced.
This enables the electricity provider to receive the electricity consumption of a household every 15 minutes. It offers the possibility to get more easily the current electricity demand from the consumers. Previously, the current electricity demand was simulated from load forecasting models. If the demand should increase spontaneously, peaker plants, mainly consisting of coal-fired power plants, would be turned on to quickly meet this demand. This is costly and environmentally unfriendly. 
Since then, structural changes have been made to optimize the energy grid and make it more intelligent by exchanging information in near-real-time. This allows the demand to be matched to the available supply. The fundamental component of the smart grid are the smart meters, which were already mentioned. They which will be discussed in more detail in the next section.(Quelle:Smart Grid Communications)(Privacy Survey2013)

\section{Smart Meter}
Smart meters are the key component in a smart grid. A smart meter is an electricity meter that has an interface to the Internet. It enables two-way communication between the control center and the meter. This is also called Advanced Metering Infrastructure (AMI). Two-way communication improves the quality of the power grid and makes it possible to offer services that would not be feasible without a smart meter. For example it's now practicable to detect power outages.%hier nochmal was ändern.
As a result, the power grid operator can detect power failures on its own. Previously, the operator was dependent on customer calls to detect power outages. Another new feature is detailed monitoring of power flows at the smart meter. Before, power flows could only be measured up to substations. This new function enables electricity network operators to quickly detect changes in consumption behavior and react to them without having to use peaker plants, which are costly and environmentally unfriendly. Depending on the setting, smart meters can send electricity consumption to the electricity provider at least every 15 minutes. In combination with the consumption of all users and the current electricity supply, a better price can be achieved. This is also called real-time pricing. So not only can the customer be offered a better electricity contract, in addition the meters no longer have to be read at home by a technician from the electricity provider. This makes billing easier for customers and electricity providers. Furthermore, customers can also check their current electricity consumption via the interfaces provided by the smart meter in order to analyze their own behavior and to reduce their consumption. (Privacy-Aware Smart Metering)
%Thus, the end user can analyze his own consumption behavior in order to reduce his own electricity consumption.
\subsection{Smart Meter Privacy}
The main advantage of the smart grid is the communication between the consumers and the energy suppliers. It is precisely this communication that solves a lot of structural problems in today's energy system. However, sending user information every 15 minutes allows for new methods that can be used to create accurate behavioral analyses in one's own home. Sending private electricity consumption data is therefore very sensitive information and must to be protected. This is not an easy task, because on the one hand the electricity consumption must be protected and anonymized, and on the other hand the billing and costs must be clearly assignable to a person. In the following paragraphs, we will describe how simple behavioral analyses are generated by electricity consumption. Subsequently, solutions to Metering for Billing and Metering for Operations will be presented, which have been discussed in the scientific community so far. \\
\\
\textbf{Nonintrusive load monitoring}\\ 
\\
\begin{figure}[tbp]
  \centering
  \includegraphics[width=1\textwidth]{images/nilm.png}
  \caption[Short description]{An example day of a NILM analysis.}
  \label{fig:Nilm}
\end{figure}



\todo{write state}

\cleardoublepage

%%% Local Variables:
%%% TeX-master: "diplom"
%%% End:
