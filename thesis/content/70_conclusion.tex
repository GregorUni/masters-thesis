\chapter{Conclusion and Future Work}
\label{sec:conclusion}

%  Schlußfolgerungen, Fragen, Ausblicke

% Dieses Kapitel ist sicherlich das am Schwierigsten zu schreibende. Es
% dient einer gerafften Zusammenfassung dessen, was man gelernt hat. Es
% ist möglicherweise gespickt von Rückwärtsverweisen in den Text, um dem
% faulen aber interessierten Leser (der Regelfall) doch noch einmal die
% Chance zu geben, sich etwas fundierter weiterzubilden. Manche guten
% Arbeiten werfen mehr Probleme auf als sie lösen. Dies darf man ruhig
% zugeben und diskutieren. Man kann gegebenenfalls auch schreiben, was
% man in dieser Sache noch zu tun gedenkt oder den Nachfolgern ein paar
% Tips geben. Aber man sollte nicht um jeden Preis Fragen, die gar nicht
% da sind, mit Gewalt aufbringen und dem Leser suggerieren, wie
% weitsichtig man doch ist. Dieses Kapitel muß kurz sein, damit es
% gelesen wird.
Smart meters offer a great opportunity to automate the existing power grid and increase the quality in the grid \ref{sec:intro}. At the same time, smart meters can be misused to create accurate behavior profiles of individuals in their own homes \ref{subsec:NILM_sec}. The BSI's technical guideline is a first standard that establishes uniform requirements for the security of smart meters and provides a minimum level of protection. Nevertheless, analyses of electricity consumption can be carried out by the electricity provider without any major effort, as pseudonymization alone cannot prevent this \ref{sec:TR_03109}. Therefore, a technical solution to the problem must be found and a variety of proposals are already being considered in the scientific community.\\ This work deals with DC networks, which are classified as aggregation without trusted third party methods. DC networks have not been considered in detail in science as a solution for privacy-preserving smart grids. The proposed design is a dc network based network protocol, which follows the requirements of the BSI technical guideline. Even though not all requirements of the technical guideline could be implemented in detail, the considered design can be adopted in the German Smart Grid without major structural changes to the existing infrastructure or software \ref{sec:design}. The solution therefore deviates from the original DC network as described e.g. in \cite{chaum1988dining}, because the restrictions of the technical guideline would not be compatible with a normal DC network and the classical DC network could not be implemented. Therefore, the electricity provider must be given administrative authority over the DC network as well as over the key graph in order for the DC network to be compliant with the BSI technical guideline. The proposed design changes had to be weighed throughout that, on the one hand, the DC network does not lose in security and anonymity. On the other hand, the proposed DC network must be efficient and operable in the real world with the administrative powers of the power provider. It must be able to dynamically respond to errors and fend off potential attackers. The two requirements of security and operability have always been at odds with each other and have meant that any change to the classic DC network has had to be carefully considered. \\
In the chapter \ref{Evaluation}, the characteristics of the proposed DC network were considered in terms of security and performance. While there may be an overhead in terms of message volume, the DC network does not lose in security compared to the classical DC network. In the case of an attacker, the power provider has procedures available to protect its network \ref{error_corr} and in the case of the strongest attacker, namely a malicious power provider, the clients have options available to protect themselves from the power provider. Furthermore, the design of the network provides many hurdles for the DC network to be structurally protected from a malicious power provider. The increased message volume is negligible since the power consumption is sent only every 15 minutes and likewise error correction actions result in only a slightly higher message volume. Moreover, the proposed DC network is scalable and can handle large numbers of users \ref{performance}.\\
In addition, the proposed DC network was implemented on four Raspberry Pis (three clients and one electricity provider) to simulate real conditions and prove that it is a workable and viable protocol. Furthermore, a stochastic experiment was implemented to study how the information content of electricity consumption is preserved in aggregated results, so that also no conclusions can be drawn from the aggregated result. The findings from the experiments were used to obtain a minimum number of DC network participants. The results show that a high level of anonymity can be achieved with as few as 10-25 participants \ref{sec:experiments}. \\
\\
\textbf{Future Work}
\\
\\
Some tasks could not be implemented in the work due to time constraints. In the case of the manipulation of the local superposition \ref{subsec:mani_local}, the algorithm could not be implemented in the demo. Since the algorithm was defined by the author and not implemented, the correctness of the algorithm cannot be answered with complete certainty. \\
Furthermore, the experiments looked for a minimum number of participants in a DC network. Different metrics were used and the results were analyzed. To get further clarity on the minimum number of DC networks, it would be possible to perform a cluster analysis and linear regression. Due to time constraints and lack of experience in the field of machine learning, these analyses could not be implemented.




%\todo{der algo wurde nicht auf korrektheit überprüft? wo das hinschreiben in Conclusion oder in evaluation?}

\cleardoublepage

%%% Local Variables:
%%% TeX-master: "diplom"
%%% End:
