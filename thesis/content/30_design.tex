\chapter{A Privacy-Preserving Aggregation Scheme Using DC-Nets}
\label{sec:design}

% Ist das zentrale Kapitel der Arbeit. Hier werden das Ziel sowie die
% eigenen Ideen, Wertungen, Entwurfsentscheidungen vorgebracht. Es kann
% sich lohnen, verschiedene Möglichkeiten durchzuspielen und dann
% explizit zu begründen, warum man sich für eine bestimmte entschieden
% hat. Dieses Kapitel sollte - zumindest in Stichworten - schon bei den
% ersten Festlegungen eines Entwurfs skizziert werden.
% Es wird sich aber in einer normal verlaufenden
% Arbeit dauernd etwas daran ändern. Das Kapitel darf nicht zu
% detailliert werden, sonst langweilt sich der Leser. Es ist sehr
% wichtig, das richtige Abstraktionsniveau zu finden. Beim Verfassen
% sollte man auf die Wiederverwendbarkeit des Textes achten.

% Plant man eine Veröffentlichung aus der Arbeit zu machen, können von
% diesem Kapitel Teile genommen werden. Das Kapitel wird in der Regel
% wohl mindestens 8 Seiten haben, mehr als 20 können ein Hinweis darauf
% sein, daß das Abstraktionsniveau verfehlt wurde.

% Die Lösung kurz vorstellen. also du hast dich für die lösung mit dem dc network entschieden und das wurde bisher noch nicht vorgeschlagen in der wissenschaft. Dann sagen bevor du das prinzip vorstellt werden erstmal alle Teilnehmer, die in dem System vorkommen, vorgestellt. dazu auch das bild aus der technischen richtlinie benutzen, wo smartmeter und smart meter admin. welche netzwerke also han etc...
%Dann sagen, dass du erstmal auf das angreifermodell eingehst und dann dc netz erklären und dann die lösung vorschlagen.
%die lösung wurde noch nicht diskutiert, weil in der wissenschaft immer davon ausgegangen wurde, dass smart meter nicht sicher sind,wir gehen davon aus, dass ein smart meter vertraut werden kann und das man deshalb auch nicht auf das billing eingehen muss, weil das smart meter korrekt arbeitet und das billing deshalb trivial ist. der stromanbieter kann auch prüfen  bzw. die logeinträge sind fälschungssicher.

%This chapter outlines the conceptual solution of this thesis to achieve privacy-preserving smart meters. The proposed protocol can be categorized as aggregation without a trusted third party[ref]. Before discussing the conceptual solution, the technical guideline from the BSI will be explained. The BSI is the cyber-security authority of the German government and is responsible for critical infrastructures such as smart grids in Germany.  The technical guideline TR-03109 resolves all security standards and security concepts that must be met by all power grid providers in Germany. Therefore, the technical guideline gives a good overview of the actual structure of the German power grid. After getting an overview of the power grid and its participants, an attacker model will be designed. The attacker model will introduce all necessary participants, what their motives are and what malicious motives they might pursue. Finally, the security protocol will be presented. It will be shown how the protocol can be integrated into the technical policy and how different potentially malicious participants are handled.

This chapter outlines the conceptual solution of this thesis to achieve privacy-preserving smart meters. The proposed protocol can be categorized as aggregation without a trusted third party[ref]. First, there will be a theoretical introduction to DC networks and an example for understanding. Subsequently, the security protocol is introduced and the functionality of the protocol is described. This includes all security mechanisms and all functions that are important for the correct execution of the protocol. In addition, mechanisms are provided to ensure the stability of the protocol in case of errors. In order to show the correctness of the protocol, the implementation of the protocol in [Ref] is presented and then the protocol is evaluated.

\section{A Privacy-Preserving Aggregation Scheme Using DC-Nets}
\todo{anderen Sektion Namen ausdenken}
In [Cha3-85, Cha8-85, Chau-88], David Chaum proposes a ''round-based'' protocol which he calls DC network. The DC network offers the possibility to achieve both sender anonymity and receiver anonymity in communication networks. The operation of the DC network is explained in the following.
\subsection{DC Networks}
The DC network uses the property that any finite alphabet can be numerated( e.g a=0, b=1 etc). If an numerated alphabet from 0 is given, then this alphabet forms an abelian group (modulo alphabet size). Because of the abelian group, simple mathematical operations like addition can be performed on the numerated letters in the alphabet.
In addition, a DC network assumes that messages are always sent that are of equal length. A participant in a DC network uses one or more keys with which it superposes the messages and one generated key is then communicated to exactly one participant. 
More precisely, each participant adds locally all key characters it generates. Then, the received keys from other participants are locally subtracted and finally, all meaningful characters (the message) that should be sent are added (modulo alphabet size) from a round. The result of the operation is distributed in the communication network and is called local superposition. 
The distributed superpositions are added together globally and the result is transmitted back to all participants. Thereby only the meaningful messages remain, because all keys are exactly added once and subtracted once in the procedure. After a global superposition is distributed a round is over and a next round starts in which messages can be sent. If a participant does not want to send a meaningful message, the participant sends an empty message. This message consists only of zeros and is superposed with the key. The empty message reflects the neutral element in this structure. If all participants have sent only empty messages, the global result is a message only containing 0 for the round. If one of all participants has sent a meaningful message, the global superposition is the message for the round. If more than one participant sent a meaningful message, then the result is the total of all sent messages and a single message from the total cannot be recovered. In the last case is also called a collision. In order to solve this problem, the collision resolution algorithm with averaging can be used in a DC network. But in this proposed protocol it is mandatory that all users send meaningful messages at the same time and that all messages form collisions with each other. In the use case of the thesis the meaningful messages are the individual electricity consumptions of each user. Therefore the global superposition is the total power consumption of all users. From the aggregated result of all electricity consumptions it is impossible to recalculate a single electricity consumption from a individual user without major attacks on the DC Network.\\
\\
\textbf{Key Exchange in DC Networks}
\\
\\
Exchanging keys to calculate the local sum can be very tedious. In addition, a different key must be exchanged for each message round. Otherwise it would be very easy to calculate the key from previously sent empty messages. Therefore, so-called pseudo-random number generators (PRNG) are commonly used. The participants share the initial values of the pseudo-random number generators with each other when they join the DC network. This can be done in the same way as the exchange of keys e.g. via a cryptographic key exchange procedure. Due to the deterministic property of PRNGs, the same number is always generated from an initial value. The result of the PRNG can be used again as a seed to produce another pseudo-random number and so forth. Therefore for each message round in a DC network a PRNG can produce a random number after only exchanging one initial value. The random numbers are used as keys for the local superposition. This in turn means that the initial value must remain secret and must not be revealed to any other participant, since otherwise the secret keys can be found out from the initial value. The consequence would be the loss of anonymity. The security of the DC network depends largely on how secure the PRNGs are. Therefore, the PRNGs that are used must be cryptographically secure.\\
\begin{figure}[tbp]
  \centering
  \includegraphics[scale=1]{images/Schlüsselgraph.png}
  \caption[DC Network Key Graph]{A basic example of a DC network key graph.}
  \label{fig:keygraph}
\end{figure}\todo{alle Bilder nochmal auflösender einfügen}
The principle of the DC network is illustrated graphically in figure \ref{fig:keygraph} using a simple example. This visualization is also called the key graph of a DC network. A key graph is the underlying graph that is created when two participants in the DC network exchange data with each other. The users form the nodes and when two users have exchanged initial values for the PRNG, they form an edge in the graph. The example DC network shows 4 participants that are connected to each other along a communication link. The outer participants have only one partner, the inner participants are connected to 2 partners. Each participant has exchanged keys with its direct partners. The outer partners needed to exchange only one key and the inner ones exchanged keys with two direct partners.
The mathematical operation indicates whether the participant adds or subtracts the exchanged key with the partner. In the example given in the figure \ref{fig:keygraph}, the user Petra would subtract the exchanged key with Rüdiger from the message she wants to send. The result is the superposition of Petra. Rüdiger would have to add the exchanged key with Petra to his message. He would also have to subtract the key he exchanged with Sabine from the result to calculate his local superposition. After everyone in the network has calculated his local superposition, all local superpositions are distributed to every user. Subsequently, every participant in the DC network can calculate the global superposition from all local superpositions.\todo{hier nochmal drüber lesen}

%peusodzufallsgeneratoren
\section{DC Network Protocol in a German Smart Grid}
The DC network is a scheme that can be used to achieve sender anonymity and receiver anonymity. Considering the use case of the thesis, the receiver anonymity does not have to be implemented. Since the aim is to anonymize the electricity consumption of a customer and send it to the electricity provider. In this case, the electricity provider is a public recipient and known to all participants. Therefore, the identity of the electricity provider does not need to be protected. Unlike in a normal DC network, in the proposed solution the participants do not want to communicate with each other, they only want to send their electricity consumption to the electricity provider. Therefore, the global superposition does not have to be distributed in the network but only has to be calculated at the electricity provider. The only exception is when joining the DC network.
During the registration process the customers have to perform a key exchange once to configure the initial value for the PRNG as explained in [ref]. Before introducing the core functions of the network protocol, it is described how the protocol can be arranged with the technical guideline of the BSI.\\
\\
\textbf{Compatibility with the Technical Guideline}
\\
\\
\begin{figure}[tbp]
  \centering
  \includegraphics[width=1\textwidth]{images/Top-down.png}
  \caption[DC Network Compatibility with TR]{
In the figure, the 6 components of the DC network frame are
are shown.}
  \label{fig:frame}
\end{figure}
\todo{In English}
The DC network must be compatible with the installed infrastructure. Otherwise, the design would not be implementable and impractical for real-world use. Therefore, a large part of the requirements from the technical guideline were taken into account in the creation phase of the design. In Figure [Ref], it can be seen how the DC network is integrated into the stakeholder model of the technical guideline. In addition, the interface boundaries of LMN and WAN are drawn. The SMGW receives the power consumption data from the meters of the different houses. The SMGW implements the communication profiles configured by the GWA. Despite the implementation of the DC network, there is no impact on the processing of the data, since the DC network is performed only after the application of the communication profiles[ref to meter processing2 siehe apendix]. The creation of the local sum is the last processing point before sending the DC network protocol data to the EMT. The local sum is then forwarded to the EMT via the GWA. Although the GWA is between the EMT and the SMGW during the transmission of the data, the GWA only forwards the data and does not perform any operations or process the protocol data. Therefore, the GWA will not be considered further in the protocol below. The power supplier\todo{Figure auf englisch} then calculates the global sum from all local sums. The design aligns with the existing stakeholder model in the technical guideline. Additionally, only the DC network protocol needs to be implemented. The exact implementation of the protocol is described starting in the next paragraph. \\
\\
\textbf{Protocol Header}
\\
\\\begin{figure}[tbp]
  \centering
  \includegraphics[width=1\textwidth]{images/Header.png}
  \caption[DC Network Frame]{
In the figure, the 6 components of the DC network data package are
are shown.}
  \label{fig:frame}
\end{figure}\todo{Protocol Identifier}In the network protocol messages are sent via data packages. The structure of a package in the DC network protocol is shown in the figure \ref{fig:frame}. Each package consists of a small header and the data part in which usually the local superposition is transfered. The purpose of each field is described further below.\\
As first field there is a Protocol Identifier field. The purpose of the Protocol Identifier field is to ensure that the application of the DC network protocol is recognized by all participants and that the SMGWs as well as the electricity provider can react correctly to the frames. This is followed by the DC Network Identifier field.%muss ich das hinschreiben?
The DC network identifier field offers the electricity provider the possibility to operate several DC networks in different regions and to distinguish DC networks. In addition, each SMGW is given a unique identifier so that the SMGWs in a network can be distinguished. If the network needs to perform error correction, SMGWs can be notified by the identification number from the power supplier. Although the SMGWs can be identified by the field, the electricity provider still cannot draw any conclusions about electricity consumption from the local superposition. The transmission bit indicates that an SMGW has sent a local superposition to the electricity provider in a round. Furthermore, it is used for error correction procedures. The time stamp field indicates when a message was sent by a participant. This allows the electricity provider to classify messages by round and not mix up for messages from different rounds. The second to last field is the notification field. The field is used for correction procedures or notifications for certain operations. An overview of all notification codes is presented in table 3.1. The electricity provider can thus send notifications to the SMGWs to start error correction procedures. In the last field, the Data field, only the local superpositions are sent. The SMGWs do not transmit any information other than the power consumption in the data field. Therefore, it can be assumed that rather small messages are sent with the proposed protocol.\\
\\
\textbf{Protocol Initialization}
\\
\\
For the Protocol Initialization it is assumed that the electricity provider wants to create a new empty DC network. First, a unique and unchangeable DC Net Identifier is assigned from the electricity provider to the empty DC Net. At least 2 SMGWs have to enter the DC Net. A DC Net with only one participant is not operational and cannot offer anonymity. 
The SMGWs that enter the network are assigned a Client Identifier by the electricity provider.\\
%vllt woander hin schreiben
To ensure a minimum level of protection for participants in the DC network, the DC network must have a minimum number of user. Even if the electricity provider receives aggregated electricity consumption, individual households may be more noticeable. Different house sizes and number of people in a household leads to a significantly higher electricity consumption, which is visible in the aggregated result for small DC networks. In this master thesis, a stochastic analysis is performed in the experiments section to determine a minimum number of participants. Moreover, it is assumed in the DC network that each participant has at least 3 connections to neighbors. This reduces the risk of individual SMGWs being disconnected from the DC network or malicious neighbors being able to reconstruct the power consumption from the local superposition.\\
According to the Technical guideline TR-03109 from BSI, SMGWs are only allowed to communicate with authorized participants in the smart grid and all foreign requests are ignored. These are EMTs, GWAs and the electricity provider. In order for the DC network to become operational, two SMGW must exchange an initial seed value to configure the PRNGs. %When a start value is exchanged, both PRNGs of the clients are initialized. 
As a result, the same random number sequences would be generated independently of each other by the PRNG on both clients. But there is a current communication barrier that does not allow SMGWs to communicate with other SMGWs. With the limited communication capabilities, the SMGWs rely on the electricity provider. The electricity provider must provide a tunnel for the SMGWs so that the SMGWs can communicate with each other and exchange the start seed for the PRNG. The tunnel works as follows. An SMGW is sent the tunnel information by the power provider. The SMGW can then send messages to the power provider to the tunnel. When messages arrive at the tunnel, the power provider forwards the messages to the receiving SMGW. Of course, the electricity provider has the technical possibility to eavesdrop on the tunnel. For this reason, secret information such as the start value of the PRNG must be cryptographically securely exchanged via the tunnel. One method to securely exchange the start value would be the Diffie Hellman key exchange.\\
\begin{figure}[tbp]
  \centering
  \includegraphics[scale=0.7]{images/key_exchange.png}
  \caption[Diffie-Hellman Key Exchange in TR-03109]{The Diffie-Hellman key exchange between two SMGWs via a electricity supplier (ES).}
  \label{fig:keyexchange}
\end{figure}
%vllt noch den tunnel einzeichnen im diagramm?
Diffie-Hellman is a known key exchange protocol, where 2 users can publicly exchange a secret over a unsecure channel without a third person being able to figure out the secret. 
Diffie-Hellman requires a generator g and a prime p, and the two values are public to all users. If two users want to exchange a key via Diffie-Hellman, then both users choose a secret random number a that is between 1 \todo{mathe}and p-1. Each user calculates the public key ka by \[ ka=g^a \]. The public key is distributed to the partner and at the same time the partner's public key \[ kb=g^b \] is obtained, where b is the random number between 1 and p-1 of the partner. After obtaining the public key, both users can calculate\[ kab=g^ab \]. Even if an attacker could eavesdrop on the values $g^a$ and $g^b$, he would not be able to compute $g^ab$ because the computation of the discrete logarithm problem cannot be performed efficiently. How the Diffie-Hellman key exchange is executed in the smart grid use case is shown in Figure \ref{fig:keyexchange}. 
%fußnote. natürlich ist in der kommunikation zwischen sm und es noch der gwa, der aber vernachlässigt werden kann, da der nur die daten weitersendet zum es.
Both SMGWs receive their public generator and prime number from the electricity provider. Then, $g^a$ and $g^b$ are calculated and sent to the partner via the electricity provider's provisioned tunnel. Afterwards, $g^ab$ can be calculated by both SMGW.\\
SMGWs can generate cryptographically secure keys because they have a hardware security module built in. Therefore key exchange procedures such as Diffie-Hellman can be executed for the SMGW without any problems. Diffie-Hellman was also only mentioned as an example. There are various attacks on the presented textbook Diffie-Hellman variant.\todo{quelle}
The forwarding of SMGW messages by the electricity provider enables the implementation of other substantially secure key exchange procedures. The advantage of this approach is that SMGWs are anonymous to other SMGWs. When the keys are exchanged, only the partners with whom the key is currently exchanged are aware of it. Uninvolved SMGWs do not receive any information about the entry of new users in a DC network. In addition, the participants share their client identifiers during the key exchange. Due to the exchanged communication details, each participant in the DC network knows the identification number of its neighbor. This is later helpful for error correction measures. %Vllt angriffe nach dem Kapitel SChreiben?
\\The use of a key exchange method also involves risks. By forwarding messages, the electricity provider knows which SMGW have exchanged keys with each other. Exchanging keys is equivalent to creating an edge in the key graph.%referenz hinzufügen als ich keygraphs erklärt habe
Therefore, a malicious electricity provider can easily perform an active attack on single SMGWs with the knowledge of the structure of the key graph.%the key graph of the DC network. The knowledge about the structure of the key graph alone does not give the electricity provider any further knowledge, but a malicious electricity provider could use the knowledge to launch active attacks on individual SMGW. 
An example would be that if a electricity provider wants to get information about the power consumption of an SMGW. The electricity provider could connect one or more SMGWs it controls to the victim SMGW through a key exchange that the attacker SMGW launches. The electricity provider could now hope that in the future the victim SMGW will only have keys with the attacker SMGWs. Since the electricity provider controls the attacker SMGW and knows the keys of the attacker SMGW, it can reconstruct the electricity consumption of the victim SMGW from the local superposition. To counter the introduced attack, participants must have a minimum number of neighbors. The more neighbors a SMGW has, the smaller the chance that each neighbor is a malicious attacker. Moreover, the electricity provider would have too much power in the DC network if it can control which SMGWs connect to each other upon entry. Therefore, a joining SMGW must be assigned to a random partner in the DC network. Furthermore, in order for error correction measures to be implemented as easily as possible, the underlying key graph in the DC network must be planar.\todo{hier nochmal drüber schauen}\\
\\
\textbf{Bootstrap Phase in a DC Network}
\\
\\
\begin{table}
\centering
\adjustbox{max width=\textwidth}{
	 \begin{tabular}{|c|c|}
	\hline
	Notification Description & Functionality\\
	\hline 
	Notification 1 & A SMGW wants to register in a DC Network\\ 
	\hline
	Notification 2 & A SMGW has succesfully entered a DC Network\\
	\hline
	Notification 3 & A SMGW is exiting a DC Network\\
	\hline
	Notification 4 & Resent local superposition according to the correction procedure\\
	\hline
	Notification 5 & Transmitting local superposition\\
	\hline
	\end{tabular}}
	\caption[Short Description]{An overview of all notification messages.} 
	\label{img:notification}
\end{table}
\begin{figure}[tbp]
  \centering
  \includegraphics[width=0.8\textwidth]{images/Registering.png}
  \caption[Sequence Diagram Registering]{
A sequence diagram showing the functions for registering an SMGW in a DC network.}
  \label{fig:sequencediagramregistering}
\end{figure}
A SMGW that wants to register in the DC network sends a special defined request to its power provider. For this purpose the notification field in the header is used and notification 1 is sent. Notification 1 represents a request from the SMGW to register in a DC network. The electricity provider assigns the requesting DC client to a suitable geographical region and suggests a random DC client (SMGW) which is already registered in the DC network. Afterwards the electricity provider establishes a tunnel and sends the tunnel information to the DC client and the registering client. Via this tunnel it is possible for the two SGMWs to create a communication link via the electricity provider. If an SMGW sends to the tunnel, the message is forwarded to its future neighbor. The DC client which is already present in the DC network is only informed by the electricity provider that it receives a new neighbor and has to exchange contact information. The requesting SMGW needs to send the seeds for the PRNG over the tunnel. To prevent the electricity provider from reading the seeds, the clients swap the seed secured by the Diffie-Helmann key exchange through the tunnels provided.%ist das korrekt?
Once the seeds have been exchanged, the power provider is informed by the requesting client that it has successfully entered the DC network by notification 2. The PRNGs are now configured and the generated keys can be added or subtracted to the message in order to create the local superposition. The procedure is explained in more detail in paragraph[ref].%erklärung zu PRNGs
The communication exchange between all participants in the registration process of a DC network is graphically illustrated in Fig \ref{fig:sequencediagramregistering}. %nochmal diagram anpassen.
Afterwards all participants in the DC network can send their local superpositions to the power provider. The provider forms the global superposition and receives the aggregated power consumption of all SMGWs in the DC network. If necessary, the electricity supplier must ensure that in the future messages can continue to be exchanged between registered customers via the same tunnel. In addition already registered clients can not choose which new communication partners they get. Furthermore, it is assumed that the electricity provider has already been authorized by the GWA. Otherwise, the SMGW would not be able to establish a connection to the supplier.\\
%(der Gedanke ist, dass es viele verschiedene kleine DC-Netze gibt, die getrennte Schlüsselgraphen besitzen, jedes DC-Netz sendet seine lokale Summe an den Stromanbieter, der alle Werte aufsummiert und die globale Summe bildet (für ein DC-Netz)
\\
\textbf{SGMW Regular Operation}
\\
\\
%wie sieht eine normale Operation aus? wie häufig wird gesendet? wofür wird das transmisson bit verwendet? und der timestampt
So far, the steps to initialize a DC grid into the already operating power grid have been explained. Next, a description is given of how the technical process takes place in the DC grid, assuming that no faults occur or corrective measures need to be taken. The SMGW transmits its electricity consumption periodically from the moment it enters the DC network. The time period is defined by the electricity provider but usually the transmission interval is 15 minutes. For the DC network, it is most practical if all SMGWs send their local superposition to the power supplier at the same time or within a short transmission interval (e.g. one minute). This can be done without problems, because according to ref 3.1 all SMGW must update their time in regular intervals with NTP servers. If an SMGW has not sent a local superposition within the transmission interval, corrective measures are implemented. The packet that a SGMW sends to the power provider is filled in as follows:
\begin{enumerate}
\item DC Net Identifier:\\
The DC net in which the SMGW is registered is entered here.
\item Client Identifier:\\ 
The assigned client identifier is sent in this field.
\item Transmission Bit:\\
This field is exactly 1 bit and is set to 1 when a local superposition is sent.
\item Time Stamp:\\
A time stamp is appended when the frame is generated.
\item Notification:\\
Notification message 5 is sent to inform the electricity provider that this message is a local superposition.
\item Data:\\
Generated local superposition is entered in the data field.
\end{enumerate}
The electricity provider processes the received frames according to the following procedure:\\
\begin{enumerate}
\item DC Network Identifier:\\
DC Network Identifier indicates to which DC net the message is processed.
\item Client Identifier:\\
The Client Identifier of the frame is stored in a memory structure. In the memory structure can be looked up later, which client has not sent a local superposition in the round.
\item Transmission Bit: Each message has a transmission bit set to 1. All transmission bits are added up and at the end of the round it can be checked whether all SMGWs have sent their local superposition. If the summed transmission bits do not correspond to the number of participants in the DC network, correction procedures must be applied.
\item Time Stamp: The power supplier can assign the message to the correct round.
\item Notfication: Notification message 5 informs the electricity supplier that the local superposition is being transmitted.
\item Data:
The local superposition in the field is added up with all other superpositions and the electricity provider gets the global superposition. This is the aggregated power consumption of all SMGWs in the DC network.
\end{enumerate}
\\
\\
\textbf{Teardown Phase in the DC Network}
\\
\\
\begin{figure}[tbp]
  \centering
  \includegraphics[width=0.8\textwidth]{images/Exit.png}
  \caption[Sequence Diagram Exiting]{A sequence diagram showing the functions for a SMGW, which wants to exit a DC network.}
  \label{fig:Exit}
\end{figure}
An exit can be caused, for example, when the customer changes the electricity provider. Then a defined message\todo{notification message 6?} is sent to the electricity provider. The electricity provider informs the neighbors of the exiting client with a notification message 3. However, to prevent the notification 3 from being misused by the power supplier, notification 3 can only be sent following the exit of an SMGW. Otherwise, a malicious electricity provider would be able to change the structure of the DC network at will. In addition to the notification message 3, the DC Client Identifier of the exiting SMGW is sent as well. This notification signals to the neighbors of the exiting SMGW that they must discard their PRNG configurations to client X and that they must not be used in the calculation of the local superposition in the next round. In order to avoid a synchronization error in the DC network, the "neighbors" must confirm to the power provider that the configuration has been discarded. Otherwise the case may occur that a SMGW continues to add the old key to its message. This would result in a useless global superposition. Furthermore, the key graph must be considered. It could happen that the underlying key graph splits into two DC networks. In this case, two separated DC networks are sending to the same DC network identifier. In the example of Figure \ref{fig:keygraph}, the scenario could occur if Sabine and Rüdiger throw away their shared key. The result is different depending on the position where a DC net splits. But at least one DC network experiences a significant loss of anonymity due to the smaller number of participants that can be aggregated. In the case of particularly serious splits, it can even lead to a participant being completely disconnected from the DC network. If a disconnected client notices that it no longer has any neighbors, it sends a special emergency message to the power provider. Then a new registration process is initiated before the next round starts.\\ To avoid splitting into two DC networks, the exiting SMGW informs its neighbors with which direct partners it was connected. These then initiate a registration process and exchange keys with each other. The fact that all neighbors have exchanged keys with each other guarantees that a DC network does not split when an SMGW leaves. Furthermore, all participants have to have a minimum number of 3 neighbors. This makes the possibility of disconnection from the DC grid much less likely, since several neighbors would have to leave the DC grid at the same time for a participant to be exposed. In the sequence diagram in figure \ref{fig:Exit} it is shown which communication exchange is performed so that an SMGW can leave the DC network.\todo{hat man dann noch einen planaren graph?}
\section{Error Correction Procedures}
It was described how DC networks work and how a DC network works in a normal operation without interference. This section explains which attacks on the network are possible and how the DC network deals with potential disturbances.
\\
\\
\textbf{SGMW Connection loss}
\\
\\
SMGW can access the Internet by communicating over their WAN interface. If the Internet connection is interrupted, this can lead to an SMGW not being able to send its local superposition in time. The result is that the electricity provider cannot calculate a meaningful global sum in the round. The electricity provider notices the error immediately because the global transmission bit does not correspond to the number of participants in the DC network. In this case, the following corrective actions are implemented: %vllt noch gliedern in 1. locate errorneous client 2. korrektur via local sum 3. save keys von client 4. restore keygraph when client is back up
The electricity provider detects which SMGW has not sent a local superposition based on the client identifier. Since the SMGW sends a complete header and the client identifier of an SMGW is also sent underneath, an electricity provider only has to check which client identifier was not sent. The missing Client Identifier is also the client that is defective. Once the defective client is located, notification 4 is sent by the power provider to the neighbors of the defective client. The notification contains the Client Identifier of the defective client and requests the neighbors to recalculate their local superposition, but without using the key of the defective client. Afterwards the updated local superposition is resent to the electricity provider. Even though the first attempt failed the electricity provider can calculate a meaningful global sum by using the resent local superposition instead of the old local superposition. At the same time, the keys of the defective client are stored by the neighbors in a backup, so that when the defective client re-enters the DC network, the same key graph is restored. The neighbors of the defective client experience no loss of anonymity during the correction process.\\%vllt noch schreiben weil jeder client mehr als eine verbindung hat
After this procedure, the defective client is temporarily no longer in the DC network. As soon as the SMGW obtains an Internet connection, it rejoins in the DC network.\todo{vllt noch schreiben wie das genauer funktioniert, das rejoinen. Also das nachbarn benachrihtigt werden, dass der defekte nachbar wieder da ist.} The power consumption of the SMGW during the time of Internet loss is not retransmitted. This is because retransmission the power consumption would lead to a complete loss of anonymity. The electricity provider knows at which time stamp which smart meter was defective and could assign the resent electricity consumption directly. Furthermore, the electricity provider knows how many smart meters are functional at any time and how many are defective through the transmission bit. Therefore the electricity provider can evaluate how conclusive the data is in the DC network. %and the assigned number of participants in the DC network. 
If not a larger amount of participants of the DC network fails the electricity provider is still able to ensure a good network stability. %Especially since the chance of an Internet outage is unlikely. 
In addition an Internet loss doesn't have an impact on the billing of a SMGW, because the calculation of the billing is executed via another procedure on the SMGW. Therefore, the electricity provider does not have to fear any loss of income, even though the electricity consumption is not sent.\\
\\ 
\textbf{Manipulation of the Local Superposition}
\\
\\
\begin{figure}[tbp]
  \centering
  \includegraphics[width=0.5\textwidth]{images/DC Net before Split.png}
  \caption[Example DC Network]{An example of a DC network before it is split in the algorithm. The edges between the nodes represent a PRNG configuration.}
  \label{fig:splitDCNetwork}
\end{figure}
One of the considerations that absolutely must be made in a DC network is: what happens if an SMGW intentionally manipulates his local superposition?
First of all, it must be mentioned that this attack does not help the customer at all to avoid the electricity costs. This is because the electricity costs are calculated by a separate procedure and therefore the billing cannot be affected by the attack. If this problem does occur, it should rather be assumed that it is an external attacker who has taken over an SMGW and wants to sabotage the availability of the DC network.
If the local superposition is manipulated, e.g. by deliberately sending a wrong local superposition, it is no longer possible for the power supplier to calculate a meaningful global superposition. Ergo, it is not possible to see the aggregated power consumption for the whole DC network. In the case of the attack, the electricity provider cannot assume that the situation will resolve itself and must take measures to find the manipulating SMGW. Furthermore, it must be assumed that the attacker is well aware that the electricity provider will be looking for him. Therefore, the procedure must be designed in such a way that the attacker is found even though he tries to conceal his identity.
For this purpose, a slightly modified version of Prof. Dr Pfitzmann's error localization and recovery protocol can be applied. The protocol of Prof. Dr Pfitzmann talks about 2 different modes, the anonymity mode (A-Mode), in which the DC net works normally and the fault tolerance mode (F-Mode), in which defective stations are searched for .
The F-mode can be extended in this application so that an attacker can also be searched for. If there is an incorrect calculation in the global superposition, this is communicated publicly by the power supplier to the SMGW. All SMGWs then save the keys from the last round. The power provider saves all receiving local sums from the round and enjoys special rights that only prevail in F-mode. The figure \ref{fig:splitDCNetwork} shows an example DC network to illustrate how the proposed algorithm works.
In the following, the property of DC networks is exploited that allows a DC network with one meaningful global superposition to separate into two DC networks with two separate meaningful global superpositions.
The algorithm is executed as follows:
1. halve the key graph. 
The power provider has the overview of the key graph and can therefore separate the key graph into two parts. Splitting the graph into two halves should be trivial since the planar separator theorem holds.
If a SMGW is exactly on the border of the bisected key graph and has a neighbor in the other half of the key graph, then this SMGW is informed as a special node by the power provider that the neighbor's key is thrown away for this computation. The temporary throwing away of the keys leads to the splitting of the key graph at that point. The power provider can request an SMGW to throw away a key only in F-mode, because throwing away a key of a neighbor immediately weakens the anonymity of an SMGW. All SMGWs in one half of the key graph now retransmit the local superposition from the last failed round. The adjacent SMGWs that threw away a key, calculate the new correct local superposition without the neighbor in the other part of the key graph. This results in all nodes sending the same message from the last round except the special nodes. This allows two global superposition to be calculated. One global superposition from the first half and one global superposition from the second half of the DC network. So the old DC network round is repeated, but in a split net to reduce the number of possible attacker SMGW. The electricity provider can check by the stored local superposition if the same local superposition is really resent and can check the correctness in a resent round. 
If the calculation of a global superposition fails in the first half of the DC network, then the attacker is located in the first half and the first half is separated again in two halves. If the calculation of a global superposition fails in the second half of the DC network, then the attacker is located in the second half and the second half is separated again in two halves. 
If the global superposition fails in both halves or is calculated correctly, then the attacker is among the special nodes.
The procedure is continued recursively until it is reduced to one SMGW that is eligible to be the attacker. The figure \ref{fig:FirstSplitting} shows the first step of the DC network splitting algorithm.
There can be the special case that the attacker SMGW is a special node. Since the special nodes send a recalculated local superposition, the power provider cannot immediately rule out whether the attacker is among the special nodes. Therefore, for each bisection, an additional subgraph must be formed in which the former special nodes have no neighbors outside the subgraph. In this way it is possible for the electricity provider to control the local superposition when resending the local sum of the former special nodes. \\
Since the power provider is granted extended rights in F-mode, it must be ensured that the power provider does not abuse F-mode by, for example, running the DC network in F-mode all the time. To prevent misuse, all SMGWs in the DC network must be informed at all times as to which mode the DC network is in. If the DC network is conspicuously often in F-mode, this could be an indication that the electricity provider is abusing its rights. In this case, the SMGWs have the option to leave the DC network and terminate the contract with the electricity provider.
\begin{figure}[tbp]
  \centering
  \includegraphics[width=0.5\textwidth]{images/DC Net Split.png}
  \caption[DC Network Splitting Algorithm]{The result of the first division of the DC network in F-mode. Red edges are division edges removed from the original network.}
  \label{fig:FirstSplitting}
\end{figure}
%was passiert mit den Sonderknoten? wie wird mit den 3 kanten umgegangen, aber man hat ja am ende nur 2 smgw?
%lsg größere gruppe mit der der angreifer identifiziert wird.
%also wenn nur noch 2 in frage kommen, werden diese beiden in eine größere gruppe gesteckt und geschaut, ob die größere gruppe falsch ist.
\\
\\
\textbf{DC Network Size}
\\
\\
With a small number of participants, conclusions can be drawn about individual participants from the aggregated result. This is the case if the electricity consumption of one user is equal in percentage to the residual consumption of the other users. Or it is also feasible that a user does not consume any electricity. In a DC network with 2 users, the power consumption would be directly readable even if the individual loads are aggregated. The goal is not to avoid the disclosure of information, but rather to make it hard to draw inferences about an individual user from the aggregated consumption[quelle].
In this thesis, the experiments chapter determines the minimum size of a DC grid to guarantee that statistical inferences are difficult to realize.
On the other hand, it is in the interest of the power supplier not to realize huge DC networks. The more participants a DC network has, the more frequently errors occur that have to be corrected by corrective measures. Although the DC network should scale with many participants, the question arises as to how meaningful the results are when hundreds of participants partially fail.%Etwas über die Größe schreiben, dass die DC netze nicht zu klein sein dürfen, aber es auch vorteile bringt, wenn sie nicht zu groß sind.
%vllt noch schreiben wie das in die technische Richtlinie implementiert werden könnte?


%noch unbedingt schreiben, dass in dem Protokoll unbedingt gewollt ist, dass es zu kollisionen kommt und das in einem normalen DC netz kollisionsauflösungsverfahren angewendet werden. siehe überlagerndes empfangen


\section{Implementation}
\label{sec:implementation}
% Hier greift man einige wenige, interessante Gesichtspunkte der
% Implementierung heraus. Das Kapitel darf nicht mit Dokumentation oder
% gar Programmkommentaren verwechselt werden. Es kann vorkommen, daß
% sehr viele Gesichtspunkte aufgegriffen werden müssen, ist aber nicht
% sehr häufig. Zweck dieses Kapitels ist einerseits, glaubhaft zu
% machen, daß man es bei der Arbeit nicht mit einem "Papiertiger"
% sondern einem real existierenden System zu tun hat. Es ist sicherlich
% auch ein sehr wichtiger Text für jemanden, der die Arbeit später
% fortsetzt. Der dritte Gesichtspunkt dabei ist, einem Leser einen etwas
% tieferen Einblick in die Technik zu geben, mit der man sich hier
% beschäftigt. Schöne Bespiele sind "War Stories", also Dinge mit denen
% man besonders zu kämpfen hatte, oder eine konkrete, beispielhafte
% Verfeinerung einer der in Kapitel 3 vorgestellten Ideen. Auch hier
% gilt, mehr als 20 Seiten liest keiner, aber das ist hierbei nicht so
% schlimm, weil man die Lektüre ja einfach abbrechen kann, ohne den
% Faden zu verlieren. Vollständige Quellprogramme haben in einer Arbeit
% nichts zu suchen, auch nicht im Anhang, sondern gehören auf Rechner,
% auf denen man sie sich ansehen kann.




% Also kurzer Überblick über die Struktur schreiben, dann auf GRPC eingehen. Wenn GRPC erklärt wurde, würde ich den DC_rounds Service vorstellen. Dann pseudoartig auf die Funktionen eingehen und ein Diagram zeichnen.
%Wo gab es probleme bei der implementation? Es gab probleme bei der synchronisation der unterschiedlichen teilnehmer.-> unterschiedliche Sekunden. Außerdem war es neu, dass mehrere nutzer auf den selben servercode "zugreifen" und das parallele programmieren. die fork nicht vergessen.
% wo gibt es unterschiede? design von grpc eingehen, man kann keine Nachrichten weiterleiten mit grpc. Nur 2 verbindungen zwischen CLients möglich.vllt noch mehr?
The conceptual approach of the DC network was presented above. Nevertheless, smart grids are a real-world system and it must be shown that the theoretical solution can be implemented in a practical environment. This section deals with the implementation of the introduced DC network protocol. It is described which technical tools were used and where implementation problems occurred.
\subsection{Structure of the Testbed}
%In the practical Implementation in this work it was tried to implement a DC network with the same requirements as de
In the practical implementation in this work it was tried to implement a DC network with the same requirements as defined in the section above. But for technical reasons, the exact same structure could not be implemented due to the framework used. If there are any deviations from the defined protocol, then these will be described and explained in this chapter.
4 Raspberry Pis are used to realize the testbed of the design, where 3 Raspberry Pis simulate the SMGW and 1 Raspberry Pi represents the electricity provider. In the following, the Raspberry Pis that represent the SMGWs are called clients and the Raspberry Pi that represents the power provider is simply referred to as the power provider. All clients have a communication link via Lan to the power provider. However, the clients do not have a physical connection to each other. As suggested in the protocol, the only way for the clients to communicate is through the power provider. After the clients join the DC network, the clients build their local superposition and send it periodically to the power provider. The electricity provider adds up the local superposition and stores the global superposition in an external text file. In a smart grid, electricity consumption is sent to the electricity provider every 15-60 minutes. Since the implementation is a demo, the sending interval for a local superposition is 10 seconds. In addition, the demo was implemented in such a way that after 4 messages a client fails and a corrective action must be taken. After that the client can re-enter the DC network. To avoid having to implement the application and the hole underlying network protocol, gRPC was used as a framework.\\
\\
\textbf{gRPC Remote Procedure Calls - gRPC}
%hier ein kleiner text
\\
\\
gRPC is an open source remote procedure call (RPC) system developed by Google since 2015. gRPC relies on a client-server structure and simplifies the construction of linked systems. With gRPC, so-called services can be defined. Each service allows to declare different functions that can communicate via a self-selected message format referred to as Protocol Buffers. Therefore on the client the functions are implemented, while the server runs the interface and processes the client requests. On the client-side is a stub that holds the same functions that are on the server. In gRPC server and client can communicate with each other even they were implemented in different programming languages. In this work, both server and client were implemented in Python. A simple application example of gRPC is shown in Figure \ref{fig:gRPC_structure}.
\begin{figure}[tbp]
  \centering
  \includegraphics[width=0.8\textwidth]{images/grpc.png}
  \caption[gRPC Framework Structure]{An overview of the structure between client and server in gRPC.}
  \label{fig:gRPC_structure}
\end{figure}\\
\\
\textbf{Protocol Buffers in gRPC}
%hier ein kleiner text
\\
\\
Protocol Buffers are used by default in gRPC and allow structured data to be serialized. %With Protocol Buffers, structured data can be specified as message formats. 
The structured data is specified via a message format in the Protocol Buffer file. Afterwards the basic source code of the protocol is automatically generated from the defined message format by executing a Protocol Buffer method. Subsequently, data can be sent from client to server through channels provided by gRPC. Listing \ref{listing3.1} shows the implementation of the protocol header by a Protocol Buffer message format. Each header field is assigned a data type and also a unique field number that determines the order of the fields.\\
%\label{listing4.1} 
\lstinputlisting[language=protobuf2, style=protobuf,caption={%
The implementation of the DC network protocol header in a Protocol Buffer message format.}]{protobuf/dc_net.proto}
%\lstinputlisting[language=protobuf2, style=protobuf,caption={%
%The implementation of the DC network protocol header in a Protocol Buffer %message format.}]{protobuf/dc_net.proto}%\label{listing4.1}
\subsection{Server Implementation}
The server implements all functions that are defined as a service in the Protocol Buffer file. The client implements most of the logic of the DC network. However, when the client accesses a service, most of the service functionality is implemented on the server. Therefore, the client prepares all necessary data and sends the data over a communication channel to the server. The data is then processed by the server and the result is communicated to the client as a response. Listing \ref{listing4.2} shows all service functions implemented on the server. %The data is then processed by the server and the result is communicated to the client as a response. Listing \ref{listing4.2} shows all service functions implemented on the server. 
Each function in the service is defined as an RPC and a name is assigned to the function. In addition, the RPC accepts a message format. For example, the addClientToDCnet function uses the message format defined in Listing \ref{listing4.1}. Afterwards, the functionality of addClienttoDCnet is implemented using gRPC framework.
\\
%\label{listing4.2}
\lstinputlisting[language=protobuf2, style=protobuf,caption={%
All server functions provided by the DC network service to the client.}]{protobuf/dc_net_server.proto}
%\lstinputlisting[language=protobuf2, style=protobuf,caption={%
%All server functions provided by the DC network service to the client.]}{protobuf/dc_net_server.proto}
%\label{listing4.2}
\\ %The functions of the service can be divided into 2 categories. First, helper functions for registration or initialization into the DC network. Second, functions in which logic of the DC net is implemented. In the following, the task of the helper functions is described first. Then the more complex functions are described, which also contain logic of the DC network.
\subsection{Client Implementation}
%hier ein kleiner text
When the in client application starts, a fork function is executed on client-side. The fork function allows a process to create a second process by duplicating the address space of the calling process. The calling process is called parent and the duplicated process is called child. In the application, the parent process takes care of the child process. If the child process crashes, the parent process ensures that the child is restarted correctly. The main logic of the DC network is found in the child process of the client. Only in the child process the services provided by the server are called and used. In addition, the communication channel of gRPC is implemented only on the child. The communication channel forms the interface in which the defined Protocol Buffers can be sent as messages. This means that the parent process has no access to the communication between server and client. Once the communication channel is configured, the interaction between client and server can begin. Figure 4.2 illustrates the structure of the client implementation in a flow chart.
%In the demo, a client is periodically forced to crash so that the error correction procedures can be demonstrated. The actual logic of the application and the communication between client and server is located in the child process. 
\\
\\
\textbf{Initialization}
\\
\\
First, it is assumed that the server is already started. Otherwise, the client cannot be started in gRPC, since no communication channel can be created. The server waits for requests from the client and processes the requests. Second, the child establishes a connection to the electricity provider. In response, the child receives a DC Net identifier and a client identifier. A registering client is stored with Client Identifier by the server in a list. This gives the server an overview of the number of participants in the DC Net. In addition, the server can identify faster which client could not send its messages in case of error correction measures. \\
\\
\textbf{Registration}
\\
\\
After a client has received a DC Net Identifier and a Client Identifier, the client requests the server to assign it to a random neighbor. In order for the neighbor and the registering client to synchronize the configuration of their PRNG, both have to perform a Diffie Hellman key exchange. All the necessary information for calculating the public key is provided by the server and is communicated to the clients on request. In the proposed protocol from the previous chapter, it was described that the electricity provider must offer a tunnel so that two SMGWs can communicate and exchange public keys via Diffie Hellman with each other. The tunnel in the protocol represents the communication channel in gRPC. But in gRPC all RPC requests are started by the client. The server only responds to the client's requests. Forwarding messages from a requesting Clients to a non-requesting clients is therefore not possible or very difficult to implement in gRPC. Instead of the public keys being forwarded to the clients by the server, as suggested in the protocol, the public keys are stored by the clients on the server for a short time and the clients make a request for the public keys. 
Each client only needs to obtain the public key of its neighbor through a request to the server and is able to compute the secret key. After the Diffie-Hellman key exchange has been executed and the PRNG has been configured, the client can create the local superposition and send its local superposition to the server for each round. 
\\
\\
\textbf{Generation of the global Superposition}
\\
\\
The server receives all the local superposition and adds them up to get the global superposition. Subsequently, the server verifies the correctness of the global superposition. If the global superposition is incorrect, then a client in a DC network has failed and could not send a local superposition. The defective client is located by the server by looking up which client has not sent a local superposition to the server. Afterwards the neighboring clients are instructed by the server to recalculate the local superposition without using the key of the failed client and resent it to the server. The global superposition is then updated by the power provider. \\
\\
\textbf{Preparations for the next Round}
\\
\\
After each sending of the local superposition, all clients check whether a new subscriber wants to register in the DC network. If so, one client is notified by the server that it gets a new neighbor. Next another registration process takes place with the Diffie-Hellman key exchange which was described before. Another deviation from the protocol is that in the implementation each client has no minimum number of neighbors than the required 3 neighbors in the protocol. The flow chart in Figure \ref{fig:Client Implementation} illustrates the structure of the client implementation with the key functions.
\begin{figure}[tbp]
  \centering
  \includegraphics[width=0.8\textwidth]{images/Client_structure.png}
  \caption[Flowchart Client Implementatioen]{The flowchart shows all program operations that are performed in the client.}
  \label{fig:Client Implementation}
\end{figure}

%\\
%\textbf{Helper Functions}
%%\\
%\\
%The helper functions in the service mainly perform tasks to register the client in the DC network or to ensure client communication with each other. Therefore, the helper functions do not take over any core logic in the DC network. In the following a documentation with the tasks of the functions is described.
%\begin{enumerate}
%\item addClientToDCnet:\\
%The first action that the client child performs is to register on the DC network. For this the addClientToDCnet function is used. In the server the request from the client is processed. The client is assigned a client identifier and the identifier is added to a list so that the service has an overview of all clients in the DC network.
%\item ConnectDCClients:\\
%The ConnectDCClients function assigns a random neighbor to a requesting client.

%\end{enumerate}
%\\
%\\
%\textbf{DC Network Functions}
%\\
%\\

\subsection{Challenges}
Several challenges were encountered during the implementation in this thesis. The most serious and time-consuming 2 errors are described below.\\
\\
\textbf{Multiple Client access on the Server}
\\
\\ 
In the experimental environment, 3 clients communicate with a server. A particular challenge was therefore to implement the server cleanly and consistently so that multiple clients could access the same function or even the same line of code at the same time without the server crashing or the program entering an inconsistent state. The implementation was therefore particularly difficult in the service functions in which the calculations of the global superposition or the verification of the global superposition were carried out.%vllt noch irgendwas schreiben mit Da der Author zuvor noch keine Erfahrung in verteilen Programmieren hatte etc.
\\
\\
\textbf{gRPC Client-to-Client Communication}
\\
\\
The protocol defined that the electricity provider must provide a tunnel for SMGW to SMGW communication. It was tried to implement the same structure as in the protocol. However, the clients in the gRPC framework do not receive an IP address, so the server cannot distinguish calls from different clients. In search of a solution the following suggestions have been considered. 
\begin{enumerate}
\item Setup a Server on each Client:\\
In this approach, each Raspberry Pi on which a client application is implemented would get an additional server that can communicate with the power provider. In addition, a client would have to be implemented on the power provider so that requests from the power provider can be sent to clients over a communication channel. In the work it was decided against it, because it is a considerable additional effort and requires an extra implementation of servers on the clients.
\item Bi-directional Streaming:\\
gRPC offers different ways to send messages. One way is a bidirectional streaming RPC where server and client exchange a sequence of messages via a stream. In gRPC can be configured depending on the use.
For example in a message stream the server can wait for all client requests before responding or it can respond immediately after each message. It can be defined in the Protocol Buffer syntax with the keyword stream. The stream property can be exploited to implement message forwarding in gRPC. Various solution sketches have been found that promote this approach. However, the proposed solutions were vague and storing the public keys in the short term on the server was much simpler to implement. 
\end{enumerate}\\
%\\
%\textbf{Crashing Parent}
%\\
%\\
%During the implementation there were indeterministic crashes of the parent in the client. The error could not be traced for a long time, because the crash occurred at different times during the runtime of the application. When the parent crashes, the child continues to work without interruptions. However, the child cannot be restarted if the parent crashes. It has been found by accident that the fork function is not executed correctly after the communication channel has been created.  
%sowas schreiben wie, dass der Mehrfachzugriff auf den Server problematisch war und schwer zu implementieren, da du noch nie mit verteilten Systemen gearbeitet hast. Dann noch schreiben, dass grpc die technischen mögichkeiten eingeschränkt hat, da es keine Client zu CLient kommunikation gab. Vllt zusammenfassen, dass die Clients einfach zu implementieren waren, aber die schwerste aufgabe der server war. Das Problem mit dem Kommunikationskanal und das der Parent die ganze zeit abgestürzt ist.

%Den Service noch beschreiben!


\section{Evaluation}
\label{Evaluation}
In this chapter the properties of the DC network are analyzed. Various characteristics are evaluated, including the performance of the process, the security and the necessary minimum size of the network.
\subsection{Performance}
Regarding the performance, it is considered to what extent the proposed network is more efficient or inefficient compared to the Technical Guideline. In addition, the message and memory volume generated by the protocol is considered.
\\
\\
\textbf{Computational Performance}
\\
\\
The proposed DC network is highly scalable. Due to the fact that on the SGMW side only a simple computation of the power consumption with a generated key through a PRNG has to be performed to form the local superposition, hardly any computational power will have to be used. Furthermore, a local superposition is sent only every 15 minutes. The computational overhead caused by the sending the local superposition is negligible for the SMGW and the protocol should be able to run without problems even on lightweight systems. It has to be considered that the used PRNG really generates random numbers. The built-in hardware security module in the SGMW offers cryptographically secure PRNG. On the side of the power provider, only simple operations need to be performed as well. Every single received local superposition of a SMGW has to be added up to form the global superposition. Addition of many thousands of summands is no problem for ordinary computers and the complexity of the calculation does not deviate from the proposal of the technical guideline, since the transmitted data has to be summed up as well.\\
In case of the error correction the transmission bit allows the power provider to immediately detect which SMGW in the network has failed. Even if several SMGW fail at the same time, it is no issue for the neighboring SMGW to resend the updated local superposition to the power provider. Afterwards, the power provider efficiently calculates the global superposition [REF procedure].\\
\textbf{Message Overhead}
\\
\\
The messages sent in the protocol can be implemented with a small header. The message field in which the local superposition are entered must always be the same size, otherwise the DC network cannot be implemented. Therefore, large data packets are not to be expected, since the local superposition require a sufficiently large message field, which will not require more than several 100 bytes. \\
\\
\textbf{Message Volume}
\\
\\
When registering a requesting SMGW in a DC network, several messages must be exchanged between the SMGW and the electricity provider as well as between the requesting SMGW and the neighbor SMGW. If a DC network round runs without errors, no additional messages are exchanged.
%In case of error correction measures, a large number of multiple messages must be sent, otherwise no meaningful global sum can be calculated and the members in a DC network must coordinate to correct the error. Considering that normally only one message is sent to the power provider every 15 minutes, it should be bearable for the power provider if there is a minimal to medium traffic volume every 15 minutes in a DC network.
However, an increased message volume is to be expected during error correction. Especially in the case of an active attacker, several phases will be run in F-mode, requiring message exchange between client and server. But it must be mentioned that most of the time in the DC network no messages are exchanged at all. Messages are only transmitted when a client enters or leaves, or when the local superposition is sent. If the local superposition fails, the system has enough time to correct the error until the next round. Therefore, an increased message volume due to the error correction can be tolerated.\\
\\
\textbf{Memory Overhead}
\\
\\
The memory overhead of SMGW in a DC network is minimal. Only the PRNG key from the neighbors need to be stored. However, the power provider needs to store the key graph. For this reason the power provider has a larger storage overhead compared to the technical guideline. Additionally, it must be assumed that the electricity provider will operate multiple DC networks simultaneously. Therefore, a medium storage overhead is expected.

\subsection{Security}
The security evaluation considers the extent to which the security objectives are better or worse protected compared to the technical guideline from the BSI. Possible attacks on the security objectives have already been explained in the Design chapter[ref] and in the related work chapter[ref].
\\
\\
\textbf{Availability}
\\
\\
Availability is one of the most important security objectives of the DC network. If an SGMW should fail, e.g. due to a missing Internet connection, then no global superposition can be formed and the functionality of the network is not possible. This is equivalent to an unavailable network for the power provider. Measures have been described to deal with this failure. If an attacker succeeds in taking over an SMGW, he can deliberately send incorrect local superposition with the intention of disrupting availability. Against this active attack, an algorithm was proposed that allows the power provider to switch to F-mode and search for the attacker and restore availability. According to this, temporary outages may occur in the DC network, but all of them can be solved in a short time by troubleshooting procedures. A degradation of availability is therefore not expected.
\\
\\
\textbf{Anonymity}
\\
\\
By using PRNGs, anonymity decreases from information-theoretically secure to complexity-theoretically secure anonymity. Nevertheless even if an SMGW is controlled by an attacker, it would not be possible for the attacker to read the power consumption of other SMGWs. This is because the attacker has no access to the global superposition. This can only be calculated by the electricity provider. With the proposed method, the attacker lacks the necessary information to launch a potential attack on the DC network\footnote[4]{Assuming the attacker does not have control over a large number of SMGW in the DC network.}. Furthermore, the attacker also has no information about the key graph. This further complicates the chances of a successful attack and to deanonymize the electricity consumption of costumers. The electricity provider is potentially the most dangerous attacker in the protocol, since the SMGWs cannot communicate with each other, they rely on the electricity provider to register in the DC network. As a result, the power provider has to take over administrative tasks and therefore possesses a lot of control. To ensure that the electricity supplier is not too powerful, its competences have been restricted. The best chance of the electricity provider to break the anonymity to its customers is that the administrative powers are abused to connect individual SMGWs to malicious neighbors. Once the electricity provider manages to link an SMGW with only malicious neighbors, the local superposition can be reconstructed and the electricity consumption is visible. This is prevented by forcing the electricity provider to select a random neighbor upon entry and not being able to remove SMGWs from the network on its own. These measures make it almost impossible for the electricity provider to affect the selection of neighbors in the DC network.
\\
\\
\textbf{Eavesdropping}
\\
\\
In the case of eavesdropping on one or more channels of SMGWs by an attacker, attackers can read but not understand payload data because the local superposition is not a meaningful message. An attacker can therefore only observe when an SMGW sends local superposition. Furthermore, the attacker can assume from an increased message volume that an error correction procedure is being used in the DC network. 
\\
\\
\textbf{Malicious Electricity Provider}
\\
\\
The strongest attacker in the DC network is an electricity provider that has malicious intent with the motivation to obtain additional privacy compromising data. Due to the advanced administrative functions available to the electricity provider, the electricity provider could be tricked into exploiting its administrative rights. Therefore, there is a fine line between granting administrative privileges to the electricity provider to maintain the DC network and restricting administrative privileges so that the electricity provider cannot breach security. Careful consideration has been given in the design to what powers are necessary for the electricity provider. However, if a malicious power provider does tamper with the DC network, the SGMW will have the ability to leave the DC network.
\\
\\
\textbf{Protocol compatibility with the TR}
\\
\\
The protocol has been designed with the structural specifications of the technical guideline. Therefore, after implementation, it could be integrated into the currently specified German smart grid without any major complications. The WAN interface of the SMGW does not need to be extended and existing communication links are not affected by the protocol. 
But the biggest weakness in the presented protocol is the electricity provider, since it knows the key graph. This design decision was made only because the technical policy does not allow direct communication between two SMGWs via the WAN interface. If this requirement were changed in the technical guideline, then a decentralized DC network could be integrated into the German smart grid. This would make attacks from a malicious electricity provider impractical.

\clearpage

%%% Local Variables:
%%% TeX-master: "diplom"
%%% End:

%%% Local Variables:
%%% TeX-master: "diplom"
%%% End:



%%% Local Variables:
%%% TeX-master: "diplom"
%%% End:
